\documentclass[line,margin]{res}

\usepackage[utf8]{inputenc}
\usepackage[pdftex]{hyperref}
\usepackage[usenames,dvipsnames]{color}

\newcommand{\hide}[1]{}

\newcounter{x}
\setcounter{x}{1}
\newcommand{\publicationnumber}{\arabic{x}\stepcounter{x}}

\makeatletter
\makeatother

\definecolor{urlcolor}{rgb}{0.1, 0.1, 0.9}

\hypersetup{ 
  colorlinks=true, pdftitle={Dr. Jérôme KUNEGIS},
  pdfauthor={Jérôme KUNEGIS}, 
  urlcolor=urlcolor
} 

\hyphenation{Schaar-schmidt Chris-toph Web-logs Mag-de-burg}
 
\begin{document}

\name{{\normalfont Dr.\ rer.\ nat.\ Dipl.-Inform.}\ Jérôme KUNEGIS {\normalfont (Complete CV)}}
\begin{resume}

\vspace{0.5cm}

\hspace{-3cm}
\begin{tabular}{lcr}
  Université de Namur    &\hspace{6.3cm}\hfill& {\tt <\href{mailto:jerome.kunegis@unamur.be}{jerome.kunegis@unamur.be}>} \\
  Rempart de la Vierge 8 &                    & {\tt \href{http://twitter.com/kunegis}{twitter.com/kunegis}} \\
  B-5000 Namur           &                    & {\tt \href{http://www.linkedin.com/in/kunegis}{linkedin.com/in/kunegis}} \\
  Belgique               &                    & {\tt \href{http://networkscience.wordpress.com/}{networkscience.wordpress.com}} \\
  & & \\
  Tel.:  +32 (0)81 72 4943
\end{tabular}
\vspace{1cm}

\section{Education}

\begin{format}
\title{l}\dates{r}\\
\employer{l}\\
\end{format}

\title{\bf Dr.\ rer.\ nat.}
\employer{\href{http://www.uni-koblenz-landau.de/}{University of Koblenz--Landau}}
\dates{2011}
\begin{position}
Thesis:
\emph{\href{https://kola.opus.hbz-nrw.de/frontdoor/index/index/docId/581}{On
    the Spectral Evolution of Large Networks}}  
\end{position}

\title{\bf Dipl.-Inform.}
\employer{\href{http://www.tu-berlin.de/}{Technical University of Berlin}}
\dates{2006}
\begin{position}
Thesis: \emph{\href{https://pdfs.semanticscholar.org/42ca/e9b00864ee09bc9bc92ef3d411e20b966a8d.pdf}{Using Integer Linear Programming for Search Results Optimization}}
\end{position}

\title{\bf Baccalauréat}
\employer{\href{http://www.fg-berlin.de/}{Lycée Français de Berlin}}
\dates{1999}
\begin{position}
  Série scientifique
\end{position}

\title{\bf Abitur}
\employer{\href{http://www.fg-berlin.de/}{Lycée Français de Berlin}}
\dates{1999}
\begin{position}
  Leistungskurse:  Mathematik, Physik
\end{position}

\title{\bf Diplôme national du brevet}
\employer{\href{http://www.fg-berlin.de/}{Lycée Français de Berlin}}
\dates{1996}
\begin{position}
  Série collège
\end{position}

\section{Positions}

\begin{format}
\title{l}\dates{r}\\
\employer{l} \\
\body\\
\end{format}

\title{\bf Postdoctoral Researcher}
\employer{\href{http://www.naxys.be/}{Namur Center for Complex Systems (naXys)}, University of Namur} 
\dates{2017--present}
\begin{position}
  \href{http://nouvelles.unamur.be/upnews.2015-10-01.8995593781}{IDEES -- L'Internet de Demain pour développer les Entreprises, l'Économie et la Société} (ERDF/FEDER -- Wallonia/Wallonie)
\end{position}

\title{\bf Postdoctoral Researcher}
\employer{\href{http://west.uni-koblenz.de/}{Institute for Web Science and Technologies}, University of Koblenz--Landau} 
\dates{2011--2016}
\begin{position}
  \href{http://www.robust-project.eu/}{ROBUST} (EU FP7), \href{http://www.socialsensor.eu/}{SocialSensor} (EU FP7), \href{http://konect.uni-koblenz.de/}{Koblenz Network Collection} (KONECT), Social Information Processing (DFG WGI), \href{http://revealproject.eu/}{REVEAL} (EU FP7).
\end{position}

\title{\bf Visiting Postdoctoral Researcher}
\employer{\href{http://www.cl.cam.ac.uk/research/srg/netos/}{Networks and Operating Systems Group}, University of Cambridge}
\dates{2013}
\begin{position}
\end{position}

\title{\bf Research Assistant}
\employer{\href{http://west.uni-koblenz.de/}{Institute for Web Science and Technologies}, University of Koblenz--Landau} 
\dates{2010--2011}
\begin{position}
  WeKnowIt (EU FP7), ROBUST (EU FP7), MULTIPLA (DFG).
\end{position}

\title{\bf Research Assistant}
\employer{\href{http://dai-labor.de/}{DAI Lab}, Technical University of Berlin}
\dates{2006--2010}
\begin{position}
Projects PIA (Personalized Information Agent), Universal Recommender
(Semantic Recommendation Engine), Connected Living (Innovationszentrum
``Vernetztes Leben''), Smart Senior (Intelligent Services for Elderly People),
UCPM (User Centric Profile Management), WebTV (Semantic TV
Recommendations), SERUM (Semantic Recommenders based on Large,
Unstructured Data).    
\end{position}

\title{\bf Student Research Assistant}
\employer{\href{http://dai-labor.de/}{DAI Lab}, Technical University of Berlin}
\dates{2002--2006}
\begin{position}
Projects Corinna und Cornelius (Speech-controlled Personal Digital
Assistant), PIA (Personalized Information Agent), Synergie
(Collaborative e-Science Platform).  
\end{position}

\section{Teaching}
\begin{itemize}
\item 
  \href{https://west.uni-koblenz.de/en/studying/courses/ss16/network-theory-and-dynamic-systems}{Network
    Theory and Dynamic Systems}, University of Koblenz--Landau, SS 2016.
\item
  \href{https://west.uni-koblenz.de/en/studying/courses/ss16/seminar}{Advanced
    Topics in Network Science}, University of Koblenz--Landau, SS 2016. 
\item
  \href{https://west.uni-koblenz.de/en/studium/lehrveranstaltungen/ws1516/grundlagen-der-datenbanken}{Introduction
    to Database Systems} (Grundlagen der Datenbanken, 
  Lecturer), University of Koblenz--Landau, WS 2015/2016.
\item 
  \href{https://west.uni-koblenz.de/en/studium/lehrveranstaltungen/ws1516/proseminar-learning-analytics}{Learning
  Analytics: Aspects of Machine Learning and Empirical Psychology in E-Learning}, (Learning
    analytics: Aspekte des Machine Learnings und empirischer Psychologie
    beim E-Learning, Proseminar), University of Koblenz--Landau, WS 2015/2016.
\item 
  \href{https://west.uni-koblenz.de/en/studium/lehrveranstaltungen/ss15/network-theory-dynamic-systems}{Network
    Theory and Dynamic Systems}, University of Koblenz--Landau, SS 2015.
\item
  \href{https://west.uni-koblenz.de/en/studium/lehrveranstaltungen/ss15/recommender-systems}{Recommender
    Systems} (Seminar), University of Koblenz--Landau, SS 2015.
\item
  \href{https://web.west.uni-koblenz.de/en/studium/lehrveranstaltungen/ws1415/gddb}{Introduction
    to Database Systems} (Grundlagen der Datenbanken, 
  Lecturer), University of Koblenz--Landau, WS 2014/2015.
\item
  \href{https://web.west.uni-koblenz.de/en/studium/lehrveranstaltungen/ws1415/forschungspraktikum}{Distributed
    Scalable Network Analysis} (Research Lab), University of
  Koblenz--Landau, WS 2014/2015. 
\item
  \href{http://www.uni-koblenz-landau.de/campus-koblenz/fb4/west/teaching/ss14/seminar}{Advanced
    Topics in Network Theory} (Seminar), University of Koblenz--Landau, SS 2014. 
\item 
  \href{http://www.uni-koblenz-landau.de/campus-koblenz/fb4/west/teaching/ss14/network-theory-and-dynamic-systems}{Network
    Theory and Dynamic Systems}, University of Koblenz--Landau, SS 2014.
\item
  \href{http://west.uni-koblenz.de/teaching/ws1314/GdDB/GdDB}{Introduction
    to Database Systems} (Grundlagen der Datenbanken, 
  Lecturer), University of Koblenz--Landau, WS 2013/2014.
\item 
  \href{https://west.uni-koblenz.de/teaching/ss13/network-theory-and-dynamic-systems}{Network
    Theory and Dynamic Systems}, University of Koblenz--Landau, SS 2013.
\item
  \href{https://west.uni-koblenz.de/teaching/ss13/proseminar-soziale-netzwerke}{Social
  Networks} (Soziale Netzwerke, Proseminar), University of Koblenz--Landau, SS 2013.
\item
  \href{https://www.uni-koblenz-landau.de/campus-koblenz/fb4/west/teaching/ws1213/datenbanken}{Introduction
    to Database Systems} (Grundlagen der Datenbanken, 
  Lecturer), University of Koblenz--Landau, WS 2012/2013.
\item
  \href{https://www.uni-koblenz-landau.de/koblenz/fb4/AGStaab/Teaching/ss12/proseminar-soziale-netzwerke}{Social
  Networks} (Soziale Netzwerke, Proseminar), University of Koblenz--Landau, SS 2012.
\item
  \href{http://www.uni-koblenz-landau.de/koblenz/fb4/AGStaab/Teaching/ws1112/Datenbanken}{Introduction
  to Database Systems} (Grundlagen der Datenbanken, Assistant), University of
  Koblenz--Landau, WS 2011/2012. 
\item 
  \href{http://www.uni-koblenz-landau.de/koblenz/fb4/institute/IFI/AGStaab/Teaching/ws1011/Datenbanken/}{Introduction
  to Database Systems} (Grundlagen der Datenbanken, Assistant), University of
  Koblenz--Landau, WS 2010/2011. 
\end{itemize}

\section{Supervised \\ Theses}
\begin{itemize}
\item Marcel Reif.  Automatische Erkennung von exakten und
  Near-Duplikaten in einer Netzwerkdatenbank, \emph{Bachelor of
    Science}, University of Koblenz--Landau, 2016.
\item Nils Geilen.  Entwicklung eines Systems zur Vorhersage von
  Nutzeraktivität auf den Diskussionsseiten der Wikipedia,
  \emph{Bachelor of Science}, University of Koblenz--Landau, 2015.
\item Jesús Cabello González.  Berechnung und Approximation von
  Kürzeste-Pfad-Statistiken in großen Netzwerken für KONECT,
  \emph{Bachelor of Science}, University of Cádiz, 2014. 
\item Julia Preusse. Analysis of the WebUni Online Student Community,
  \emph{Dipl.-Inform.}, Otto von Guericke University Magdeburg, 2010.  
\item Stephan Spiegel.  A Hybrid Approach to Recommender Systems based
  on Matrix Factorization, \emph{Dipl.-Inform.}, Technical University of
  Berlin, 2009.  
\item Christian Banik.  Recommending Wiki Articles using Collaborative
  Filtering, \emph{Dipl.-Inform.}, Technical University of Berlin, 2009. 
\item Iris Breddin.  Untersuchung der Klassifizierbarkeit und
  Klassifikation von Schweißnahtdaten, \emph{Dipl.-Inform.}, Technical
  University of Berlin, 2008.    
\end{itemize}

\section{Conference Organization}
\begin{itemize}
\item \href{https://sites.google.com/site/ocm2016/}{WeST Off-Campus
  Meeting} (OCM), 2016, Chair. 
\item \href{http://informatik2013.de/}{INFORMATIK}, 2013, Publicity Chair. 
\item \href{http://www.websci11.org/}{Web Sci.\ Conf.}, 2011,
  Publicity Chair.  
\item \href{http://www.dai-labor.de/unm2010/}{Special Session on
  Uncertainty in Netw. Min.} (UNM) at the Int.\ Conf.\ on Information
  Processing and Management of Uncertainty in Knowledge-based Systems (IPMU),
  2010, Technical Chair.   
\end{itemize}

\section{Program Committees}
\begin{itemize}
\item \href{http://complexnetworks.org/index2016.html}{Int.\ Workshop on
  Complex Netw. and Their Applications}, 2016.  
\item \href{http://www.icwsm.org/2016/index.php}{Int.\ AAAI Conf.\ on
  Web and Soc.\ Media} (ICWSM), 2016.
\item \href{http://www.snow-workshop.org/}{Workshop on Soc.\ News on
  the Web} (SNOW) at the World Wide Web Conf.\ (WWW), 2016. 
\item \href{http://events.kmi.open.ac.uk/salsa2016/}{Workshop on Soc.\ Semantic Analysis} (SALSA), 2016.
\item \href{https://failworkshops.wordpress.com/fail-at-ir16/}{\#FAIL!
  -- The Workshop Series} at the Internet Research Conf.\ (IR), 2015.
\item \href{http://www.www2015.it/call-for-web-science-track/}{Web
  Sci.\ Track}, Int.\ World Wide Web Conf.\ (WWW), 2015.
\item \href{https://failworkshops.wordpress.com/fail-workshop-at-websci15/}{\#FAIL! -- The Workshop Series} at the Web Sci.\ Conf.\ (WebSci), 2015.
\item \href{http://www.websci14.org/}{Web Sci.\ Conf.},
  2014. 
\item \href{http://www.snow-workshop.org/}{Workshop on Soc.\ News on
  the Web} (SNOW) at the World Wide Web Conf.\ (WWW), 2014.
\item \href{http://www.cool2014.com/}{Workshop on Connecting Online \&
  Offline Life} (COOL) at the World Wide Web Conf.\ (WWW), 2014. 
\item \href{http://www2014.kr/calls/call-for-web-science-track/}{Web
  Sci.\ Track}, Int.\ World Wide Web Conf.\ (WWW), 2014.
\item \href{http://ijcai13.org/}{Int.\ Joint Conf.\ on Artif.\ Intell.} (IJCAI), 2013.  
\item \href{http://www.websci13.org/}{Web Sci.\ Conf.}, 2013. 
\item \href{http://webscience-education-workshop.blogs.usj.edu.lb/}{Web
  Sci.\ Education Workshop} at the Web Sci.\ Conf., 2013. 
\item \href{http://www.ftsm.ukm.my/stair13/}{Conf.\ on Semantic
  Technology and Information Retrieval} (STAIR), 2013. 
\item \href{http://spim-workshop.org/}{Int.\ Workshop on Semantic
  Personalized Information Management} (SPIM) at the Int.\ Conf.\ on Web
  Search and Data Mining (WSDM), 2013.  
\item \href{http://ecir2013.org/}{Eur.\ Conf.\ on Information
  Retrieval} (ECIR), 2013.  
\item \href{http://mama.west.uni-koblenz.de/}{Workshop on Metrics,
  Analysis and Tools for Online Community Management} (MAMA), at
  INFORMATIK, 2013
\item \href{http://www.websci12.org/}{Web Sci.\ Conf.}, 2012. 
\item \href{http://ecir2012.upf.edu/}{Eur.\ Conf.\ on Information
  Retrieval} (ECIR), poster track, 2012.  
\item \href{http://www.oegai.at/konvens2012/}{Conf.\ on Natural Language
  Processing} (KONVENS), 2012. 
\item \href{http://spim-workshop.org/}{Workshop on Personalized
  Information Management: Linking Soc.\ and Semantic Web} (SPIM) at the
  Int.\ Conf.\ on Web Engineering (ICWE), 2012. 
\item \href{http://www.cikm2011.org/}{Conf.\ on Information and Knowledge
  Management} (CIKM), 2011.
\item \href{http://www.websci11.org/}{Web Sci.\ Conf.}, 2011.
\item \href{http://www.sigkdd.org/kdd/2011/}{Int.\ Conf.\ on 
  Knowledge Discovery and Data Mining} (KDD), research track, 2011.
\item \href{http://www.dai-labor.de/camra2010/}{Challenge on
  Context-aware Movie Recommendation} (CAMRa) at the Conf.\ on  
  Recommender Systems (RecSys), 2010.   
\item \href{http://www.dai-labor.de/unm2010/}{Special Session on Uncertainty in Netw. Min.} (UNM) at the
  Int.\ Conf.\ on Information Processing and Management of
  Uncertainty in Knowledge-based Systems (IPMU), 2010. 
\end{itemize}

\section{Review Boards and Other Reviewing}
\begin{itemize}
\item \href{http://journals.cambridge.org/action/displayJournal?jid=NWS}{Netw. Sci.}, 2015, 2016.
\item \href{http://surveys.acm.org/}{ACM Comput.\ Surveys}, 2014, 2015. 
\item \href{http://www.webscience-journal.net}{The J.\ of Web Sci.}, 2014. 
\item \href{http://toit.acm.org/}{ACM Trans.\ on Internet
  Technologies} (TOIT), 2014.  
\item \href{http://www.ii.pwr.wroc.pl/~krol/eng_EPP.htm}{Special
  Issue on ``Propagation Phenomenon in Complex Networks:  Theory and
  Practice''}, New Generation Comput., 2014. 
\item \href{http://iswc2013.semanticweb.org/}{Int.\ Semantic Web Conf.}
  (ISWC), 2013.  
\item \href{http://www.compstat2012.org/}{Int.\ Conf.\ on Comput.\ Stat.} (COMPSTAT), 2012.  
\item
  \href{http://www.journals.elsevier.com/computer-networks/}{Int.\ J.\ of
    Comput.\ and Telecommunications Netw.} (COMNET), 2012. 
\item \href{http://ieee-cis.org/pubs/tnn/}{IEEE Trans.\ on Neural
  Netw.} (TNN), 2011.   
\item
  \href{http://www.springer.com/computer/ai/journal/10458}{J.\ on Autonomous
    Agents and Multi-agent Systems} (AAMAS), Special Issue on Agent Mining,
  2011. 
\item \href{http://tkdd.cs.uiuc.edu/}{ACM Trans.\ on Knowledge
  Discovery from Data} (TKDD), 2011.  
\item \href{http://mklab.iti.gr/tvca/}{TV Content Analysis} (TVCA), 2011.
\item \href{http://www.loa-cnr.it/ontolex2010}{Workshop on Ontologies
  and Lexical Resources} (OntoLex) at the Int.\ Conf.\ on Comput.\ Linguist.\ (COLING), 2010. 
\end{itemize}

\section{Articles}           \input{list-article}
\section{Conference Papers}  \input{list-conf}
\section{Workshop Papers}    \input{list-workshop}
\section{Abstracts}          \input{list-abstract}
\section{Miscellaneous Peer-Reviewed Publications}	\input{list-peer}
\section{Other Publications} \input{list-other}

\section{Keynotes and Invited Talks}
\begin{itemize}
\item[{[T1]}] Algebraic Graph-theoric Measures of Conflict, 
  \href{http://jgss.sciencesconf.org/}{Journée Graphes et Systèmes
    Sociaux} (Seminar on Graphs and Soc.\ Systems), 2016.  
\item[{[T2]}] Measuring Conflict in Signed Soc.\ Networks, 
  Application of Netw. Theory on Comput.\ Soc.\ Sci.\ (Workshop), 2015.
\item[{[T3]}] Large Network Collections:  The Power of Many Datasets,
  \href{http://www.ars15.unisa.it/}{Int.\ Workshop on Soc.\ Netw. Analysis} (ARS), 2015. 
\item[{[T4]}] Network Analysis Tools for Online Communities: The Koblenz Network
  Collection. Keynote, \href{http://mama.west.uni-koblenz.de/}{Workshop
    on Metrics, Analysis and Tools for Online Community Management}
  (MAMA), 2013.  
\end{itemize}

\section{Other Talks}
\begin{itemize}
  \item[{[T5]}] Generating Networks with Realistic Properties Based on a
    Given (Set of) Network(s), and a Short Overview of the KONECT
    Project.  Université de Namur, 2016. 
  \item[{[T6]}] Web Science in Practice:  Web Observatories.  WSTNet Web
    Sci.\ Summer Sch., 2016.  
  \item[{[T7]}] KONECT: The Koblenz Network Collection -- Towards a Broad
    Analysis of Complex Systems.  ETH Zürich, 2015.
  \item[{[T8]}] Observing the Web: The Koblenz Network Collection.
    Bournemouth University, 2013.
  \item[{[T9]}] Growth and Decay: Using Machine Learning to Predict the Web's
    Future, Workshop on Artif.\ Intell. on the Web, 2012. 
  \item[{[T10]}] Models of Like, Dislike, Similarity and Dissimilarity using
    Split-complex Numbers, University College Dublin, 2012. 
  \item[{[T11]}] Why Is Beyoncé More Popular Than Me:  Fairness, Diversity and
    Other Measures of Network Equality. University of Freiburg, 2012. 
  \item[{[T12]}] Fairness on the Web: Alternatives to the Power
    Law. 
    Leibniz-Institut für Sozialwissenschaften, Cologne, 2012.  
  \item[{[T13]}] On the Spectral Evolution of Large Networks.  University College
    Cork, 2011.   
  \item[{[T14]}] On the Spectral Evolution of Large Networks.  Fraunhofer IAIS,
    St.\ Augustin, 2011.  
  \item[{[T15]}] On the Spectral Evolution of Large Networks. Technical
    University of Berlin, 2011.  
  \item[{[T16]}] The Slashdot Zoo: Mining a Social Network with Negative
    Edges. Data and Knowledge Engineering Research Colloquium,
    Otto von Guericke University Magdeburg, 2010.  
  \item[{[T17]}] The Slashdot Zoo: Mining a Social Network with Negative
    Edges. University of Koblenz--Landau, 2010. 
  \item[{[T18]}] The Slashdot Zoo: Mining a Social Network with Negative
    Edges. University of Hannover, 2010. 
  \item[{[T19]}] PIA+COMM~-- an Intelligent Search Engine.
    Michael Hahne, Corinna Jung, Jérôme Kunegis, Andreas Lommatzsch and
    André Paus. 
    Workshop on the Formation of Soc. Netw. in Social Software
    Applications, INFORMATIK, 2006. 
\end{itemize}

\section{Talks with Co-authorship Credit}
\begin{itemize}
  \item[{[T20]}] Linguistic Neighbourhoods: Explaining Cultural Borders on
    Wikipedia Through Multilingual Co-editing Activity.  Anna
    Samoilenko, Fariba Karimi, Daniel Edler, Jérôme Kunegis and Markus
    Strohmaier. 
    Int.\ Conf.\ on Comput.\ Soc.\ Sci., 2017. 
  \item[{[T21]}] Challenges in Mining Social Media: Sparsity and Quality.
    Thomas Gottron, Nasir Naveed, Jérôme Kunegis, Arifah Che Alhadi.
    Challenges in Document Mining (Dagstuhl Seminar 11171), 2011. 
\end{itemize}

\section{Posters without Publication}
\begin{itemize}
  \item[{[P1]}] Linguistic Neighbourhoods:  Explaining Cultural Borders on
    Wikipedia through Multilingual Co-editing Activity.  Anna
    Samoilenko, Fariba Karimi, Daniel Edler, Jérôme Kunegis, Markus
    Strohmaier, Int.\ Sch.\ and Conf.\ on Netw. Sci.\ (NetSciX),
    2016. 
  \item[{[P2]}]
    Social Network Observatory.  Jérôme Kunegis, Markus Strohmaier,
    Steffen Staab.  Computat.\ Soc.\ Sci.\ Winter Symposium (CSSWS), 2015.
  \item[{[P3]}] 
    Quantifying Cultural Similarity through Language Co-occurrences
    in Wikipedia Editing Activity. Anna Samoilenko, Fariba Karimi,
    Daniel Edler, Jérôme Kunegis, Markus Strohmaier.  Comput.\ Soc.\ Sci.\ Winter Symposium (CSSWS), 2015.
  \item[{[P4]}] 
    Polarisation in Voting Platforms: A Case Study of LiquidFeedback
    in the German Pirate Party. Manuel Mittler, Christoph Carl Kling,
    Jérôme Kunegis, Markus Strohmaier, Comput.\ Soc.\ Sci.\ Winter Symposium (CSSWS), 2015.
  \item[{[P5]}]
    Twitter as a Political Network -- Predicting the Following and
    Unfollowing Behavior of {German} Politicians. Julia Perl, Claudia
    Wagner, Jérôme Kunegis, Steffen Staab, Web Sci.\ Conf., 2015. 
  \item[{[P6]}] 
    Online Delegative Democracy: A Case Study of the German Pirate
    Party's Voting Platform.  Christoph Carl Kling, Jérôme Kunegis,
    Heinrich Hartmann.  Comput.\ Soc.\ Sci.\ Winter
    Symposium (CSSWS), 2014.
  \item[{[P7]}]
    Social Networking By Proxy: A Case Study of Catster, Dogster and
    Hamsterster. Comput.\ Soc.\ Sci.\ Winter
    Symposium (CSSWS), 2014.
  \item[{[P8]}] 
    A Theory-Driven Approach for Link and Unlink Predictions in
    Directed Social Networks. Julia Perl, Claudia Wagner, Jérôme
    Kunegis, Steffen Staab.  Comput.\ Soc.\ Sci.\ Winter
    Symposium (CSSWS), 2014.
  \item[{[P9]}]
    Need Networks? KONECT -- The Koblenz Network Collection.
    Eur.\ Summer Sch.\ on Information Retrieval (ESSIR), 2011. 
  \item[{[P10]}]
    KONECT -- The Koblenz Network Collection. Web Sci.\ Conf., 2011. 
  \item[{[P11]}]
    Uncovering Multi-modal Spread Modes using Joint 
    Diagonalization in Dynamic Human Contact Networks. 
    Damien Fay, Jérôme Kunegis, Eiko Yoneki.  Interdisciplinary
    Workshop on Information and Decision in Soc. Netw., 2011. 
  \item[{[P12]}]
    On the Spectral Evolution of Large Networks. Postersession 
    für Nachwuchswissenschaftler/innen, University of Koblenz--Landau,
    2011. 
  \item[{[P13]}]
    On the Spectral Evolution of Large Networks. SIAM 
    Conf.\ on Data Mining Doctoral Forum (SIAM SDM), 2010. 
\end{itemize}

\section{Demonstrations}
\begin{itemize}
 \item[{[D1]}] Institute for Web Science and Technologies, conference booth.
       \href{http://informatik2013.de}{INFORMATIK 2013}.  
 \item[{[D2]}] Schülerinformationstage, Institute for Web Science and
       Technologies, University of Koblenz--Landau, 2011, 2013, 2015, 2016.  
 \item[{[D3]}] LiveTweet: Monitoring and Predicting Interesting Microblog
       Posts, \href{http://ecir2012.upf.edu/}{Eur.\ Conference on Information Retrieval} (ECIR),
       2012. 
 \item[{[D4]}] Time-aware Centrality in Contact Network Analysis.
       Eiko Yoneki, Damien Fay, Jérôme Kunegis.
       \href{http://www.eccs2011.eu/}{Eur.\ Conf.\ on Complex Systems} (ECCS), 2011. 
 \item[{[D5]}] PIA, Spree, CeBIT, 2010.
 \item[{[D6]}] PIA+COMM, CeBIT, 2007.
 \item[{[D7]}] PIA, Lange Nacht der Wissenschaften Berlin/Potsdam, 2006. 
\end{itemize}

\section{Mentions in Media}
\begin{itemize}
  \item
    \href{https://www.piratenpartei.de/2015/05/31/die-abstimmungssoftware-liquidfeedback-der-piratenpartei-wegweisend-fuer-demokratie-2-0/}{Die
      Abstimmungssoftware LiquidFeedback in der Piratenpartei:
      wegweisend für Demokratie 2.0?}, www.piratenpartei.de, May 2015. 
\item
  \href{http://www.wired.it/attualita/2015/04/01/come-funzionato-davvero-democrazia-liquida-dei-pirati-tedeschi/}{Come
    ha funzionato davvero la democrazia liquida dei Pirati tedeschi},
  Wired Italia, April 2015. 
\end{itemize}

\section{Awards and Nominations}
\begin{itemize}
  \item First Prize, Int.\ Conf.\ on Web and Soc.\ Media (ICWSM) Science Slam, 2016.
  \item Best Poster Award, Int.\ Sch.\ and Conf.\ on Netw. Sci.
    (NetSciX), 2016. 
  \item Best Paper Honorable Mention, Int.\ Conf.\ on Web and Soc.\ Media (ICWSM), 2015.  
  \item Ted Nelson Newcomer Award Nominee, Conf.\ on Hypertext and
    Social Media (HT), 2012.   
  \item Student Travel Award, Int.\ Conf.\ on Information and Knowledge
    Management (CIKM), 2010. 
  \item Student Travel Fellowship Award, SIAM Conf.\ on Data Mining (SDM),
    2010.  
  \item Best Paper Nominee, Industrial Conf.\ on Data Mining (IndCDM), 2007.    
  \item Premier Prix (First Prize), Rallye mathématique d'Alsace, Terminale (Alsace Mathematical Rally), 1999. 
\end{itemize}

\section{Main Software Authorship}
\begin{itemize}
  \item \href{https://github.com/kunegis/stu}{Stu}, build automation (C++11)
  \item \href{https://github.com/kunegis/konect-toolbox}{KONECT Toolbox}, network analysis toolbox (Matlab, C99)
  \item \href{https://west.uni-koblenz.de/Research/systems/polcovar}{Polcovar}, software for counting patterns in random graphs (Matlab)
  \item \href{https://github.com/kunegis/universal-recommender}{Universal Recommender}, recommendation library (Java)
  \item \href{https://github.com/kunegis/babychess}{BabyChess}, chess engine and GUI (C++98)
  \item \emph{More:  see {\tt \href{https://github.com/kunegis}{github.com/kunegis}}}
\end{itemize}

\section{Miscellaneous}
\begin{itemize}
  \item Birthday:  March 27
  \item Languages: French (first language), German (first language),
    English (advanced)
  \item Programming languages and technologies I use extensively:
    Bourne shell, C, C++, Java, Latex, Make, Matlab, GNU Octave, sed
  \item Principal other programming languages and technologies I have used:
    BASIC, HTML, JavaScript, Logo, Perl, PHP, Turbo Pascal 
\end{itemize}

\end{resume}
\end{document}
