\documentclass[line,margin]{res}

\usepackage[utf8]{inputenc}
\usepackage[pdftex]{hyperref}
\usepackage[usenames,dvipsnames]{color}

\newcommand{\hide}[1]{}

\makeatletter
\renewenvironment{thebibliography}[1]
     {\section{\bibname}% <-- this line was changed from \chapter* to \section*
      \@mkboth{\MakeUppercase\bibname}{\MakeUppercase\bibname}%
      \list{\@biblabel{\@arabic\c@enumiv}}%
           {\settowidth\labelwidth{\@biblabel{#1}}%
            \leftmargin\labelwidth
            \advance\leftmargin\labelsep
            \@openbib@code
            \usecounter{enumiv}%
            \let\p@enumiv\@empty
            \renewcommand\theenumiv{\@arabic\c@enumiv}}%
      \sloppy
      \clubpenalty4000
      \@clubpenalty \clubpenalty
      \widowpenalty4000%
      \sfcode`\.\@m}
     {\def\@noitemerr
       {\@latex@warning{Empty `thebibliography' environment}}%
      \endlist}
\makeatother

\definecolor{urlcolor}{rgb}{0.1, 0.1, 0.9}

\hypersetup{ 
  colorlinks=true, pdftitle={Dr. Jérôme Kunegis},
  pdfauthor={Jérôme Kunegis}, 
  urlcolor=urlcolor
} 

\begin{document}

\name{Dr.\ Jérôme Kunegis
%- (Complete CV)
}
\begin{resume}

\vspace{0.5cm}

\hspace{-3cm}
\begin{tabular}{lcr}
%%  University of Koblenz--Landau               &\hspace{4.0cm}\hfill& Tel: {\tt +49 261 287-2775}\\
%%   && Fax: {\tt +49 261 287-100-2775}\\
  Rübenacher Str.\ 78                       & \hspace{7.0cm}\hfill &  {\tt kunegis@gmail.com}\\
  56070 Koblenz                              && 
  %% {\tt \href{http://www.uni-koblenz.de/~kunegis/}{uni-koblenz.de/\~{ }kunegis}}
  \\
  Germany                                    &&
  {\tt \href{http://twitter.com/kunegis}{twitter.com/kunegis}} \\
  && {\tt \href{http://www.linkedin.com/in/kunegis}{linkedin.com/in/kunegis}} \\
  && {\tt \href{http://networkscience.wordpress.com/}{networkscience.wordpress.com}} 
\end{tabular}
\vspace{1cm}


\section{Education}

\begin{format}
\title{l}\dates{r}\\
\employer{l}\\
\end{format}

\title{\bf Dr.\ rer.\ nat.}
\employer{\href{http://www.uni-koblenz-landau.de/}{University of Koblenz--Landau}}
\dates{2011}
\begin{position}
Thesis:
\emph{\href{http://userpages.uni-koblenz.de/~kunegis/paper/kunegis-phd-thesis-on-the-spectral-evolution-of-large-networks.pdf}{On
    the Spectral Evolution of Large Networks}}  
\end{position}

\title{\bf Dipl.-Inform.}
\employer{\href{http://www.tu-berlin.de/}{Technical University of Berlin}}
\dates{2006}
\begin{position}
Thesis: \emph{\href{http://userpages.uni-koblenz.de/~kunegis/paper/kunegis-diploma-thesis-using-integer-linear-programming-for-search-results-optimization.pdf}{Using Integer Linear Programming for Search Results Optimization}}
\end{position}

\title{\bf Baccalauréat}
\employer{\href{http://www.fg-berlin.de/}{Lycée Français de Berlin}}
\dates{1999}
\begin{position}
%- Série scientifique
\end{position}

\title{\bf Abitur}
\employer{\href{http://www.fg-berlin.de/}{Lycée Français de Berlin}}
\dates{1999}
\begin{position}
%- Leistungskurse:  Mathematik, Physik
\end{position}

\title{\bf Diplôme national du brevet}
\employer{\href{http://www.fg-berlin.de/}{Lycée Français de Berlin}}
\dates{1996}
\begin{position}
%- Série collège
\end{position}

\section{Positions}

\begin{format}
\title{l}\dates{r}\\
\employer{l} \\
%- \body\\
\end{format}

\title{\bf Postdoctoral Researcher}
\employer{\href{http://www.naxys.be/}{Namur Center for Complex Systems}, University of Namur} 
\dates{2017--present}
\begin{position}
\end{position}

\title{\bf Postdoctoral Researcher}
\employer{\href{http://west.uni-koblenz.de/}{Institute for Web Science and Technologies}, University of Koblenz--Landau} 
\dates{2011--2016}
\begin{position}
%- \href{http://www.robust-project.eu/}{ROBUST} (EU FP7), \href{http://www.socialsensor.eu/}{SocialSensor} (EU FP7), \href{http://konect.uni-koblenz.de/}{Koblenz Network Collection} (KONECT), Social Information Processing (DFG WGI), \href{http://revealproject.eu/}{REVEAL} (EU FP7).
\end{position}

\title{\bf Visiting Postdoctoral Researcher}
\employer{\href{http://www.cl.cam.ac.uk/research/srg/netos/}{Networks and Operating Systems Group}, University of Cambridge}
\dates{2013}
\begin{position}
\end{position}

\title{\bf Research Assistant}
\employer{\href{http://west.uni-koblenz.de/}{Institute for Web Science and Technologies}, University of Koblenz--Landau} 
\dates{2010--2011}
\begin{position}
%- WeKnowIt (EU FP7), ROBUST (EU FP7), MULTIPLA (DFG).
\end{position}

\title{\bf Research Assistant}
\employer{\href{http://dai-labor.de/}{DAI Lab}, Technical University of Berlin}
\dates{2006--2010}
\begin{position}
%- Projects PIA (Personalized Information Agent), Universal Recommender
%- (Semantic Recommendation Engine), Connected Living (Innovationszentrum
%- ``Vernetztes Leben''), Smart Senior (Intelligent Services for Elderly People),
%- UCPM (User Centric Profile Management), WebTV (Semantic TV
%- Recommendations), SERUM (Semantic Recommenders based on Large,
%- Unstructured Data).    
\end{position}

\title{\bf Student Research Assistant}
\employer{\href{http://dai-labor.de/}{DAI Lab}, Technical University of Berlin}
\dates{2002--2006}
\begin{position}
%- Projects Corinna und Cornelius (Speech-controlled Personal Digital
%- Assistant), PIA (Personalized Information Agent), Synergie
%- (Collaborative e-Science Platform).  
\end{position}

\section{Teaching}
\begin{itemize}
\item 
  \href{https://west.uni-koblenz.de/en/studying/courses/ss16/network-theory-and-dynamic-systems}{Network
    Theory and Dynamic Systems}, University of Koblenz--Landau, SS 2016.
\item
  \href{https://west.uni-koblenz.de/en/studying/courses/ss16/seminar}{Advanced
    Topics in Network Science}, University of Koblenz--Landau, SS 2016. 
\item
  \href{https://west.uni-koblenz.de/en/studium/lehrveranstaltungen/ws1516/grundlagen-der-datenbanken}{Introduction
    to Database Systems} (Grundlagen der Datenbanken, 
  Lecturer), University of Koblenz--Landau, WS 2015/2016.
\item 
  \href{https://west.uni-koblenz.de/en/studium/lehrveranstaltungen/ws1516/proseminar-learning-analytics}{Learning
  Analytics: Aspects of Machine Learning and Empirical Psychology in E-Learning}, (Learning
    analytics: Aspekte des Machine Learnings und empirischer Psychologie
    beim E-Learning, Proseminar), University of Koblenz--Landau, WS 2015/2016.
\item 
  \href{https://west.uni-koblenz.de/en/studium/lehrveranstaltungen/ss15/network-theory-dynamic-systems}{Network
    Theory and Dynamic Systems}, University of Koblenz--Landau, SS 2015.
\item
  \href{https://west.uni-koblenz.de/en/studium/lehrveranstaltungen/ss15/recommender-systems}{Recommender
    Systems} (Seminar), University of Koblenz--Landau, SS 2015.
\item
  \href{https://web.west.uni-koblenz.de/en/studium/lehrveranstaltungen/ws1415/gddb}{Introduction
    to Database Systems} (Grundlagen der Datenbanken, 
  Lecturer), University of Koblenz--Landau, WS 2014/2015.
\item
  \href{https://web.west.uni-koblenz.de/en/studium/lehrveranstaltungen/ws1415/forschungspraktikum}{Distributed
    Scalable Network Analysis} (Research Lab), University of
  Koblenz--Landau, WS 2014/2015. 
\item
  \href{http://www.uni-koblenz-landau.de/campus-koblenz/fb4/west/teaching/ss14/seminar}{Advanced
    Topics in Network Theory} (Seminar), University of Koblenz--Landau, SS 2014. 
\item 
  \href{http://www.uni-koblenz-landau.de/campus-koblenz/fb4/west/teaching/ss14/network-theory-and-dynamic-systems}{Network
    Theory and Dynamic Systems}, University of Koblenz--Landau, SS 2014.
\item
  \href{http://west.uni-koblenz.de/teaching/ws1314/GdDB/GdDB}{Introduction
    to Database Systems} (Grundlagen der Datenbanken, 
  Lecturer), University of Koblenz--Landau, WS 2013/2014.
\item 
  \href{https://west.uni-koblenz.de/teaching/ss13/network-theory-and-dynamic-systems}{Network
    Theory and Dynamic Systems}, University of Koblenz--Landau, SS 2013.
\item
  \href{https://west.uni-koblenz.de/teaching/ss13/proseminar-soziale-netzwerke}{Social
  Networks} (Soziale Netzwerke, Proseminar), University of Koblenz--Landau, SS 2013.
\item
  \href{https://www.uni-koblenz-landau.de/campus-koblenz/fb4/west/teaching/ws1213/datenbanken}{Introduction
    to Database Systems} (Grundlagen der Datenbanken, 
  Lecturer), University of Koblenz--Landau, WS 2012/2013.
\item
  \href{https://www.uni-koblenz-landau.de/koblenz/fb4/AGStaab/Teaching/ss12/proseminar-soziale-netzwerke}{Social
  Networks} (Soziale Netzwerke, Proseminar), University of Koblenz--Landau, SS 2012.
\item
  \href{http://www.uni-koblenz-landau.de/koblenz/fb4/AGStaab/Teaching/ws1112/Datenbanken}{Introduction
  to Database Systems} (Grundlagen der Datenbanken, Assistant), University of
  Koblenz--Landau, WS 2011/2012. 
\item 
  \href{http://www.uni-koblenz-landau.de/koblenz/fb4/institute/IFI/AGStaab/Teaching/ws1011/Datenbanken/}{Introduction
  to Database Systems} (Grundlagen der Datenbanken, Assistant), University of
  Koblenz--Landau, WS 2010/2011. 
\end{itemize}

\section{Supervised \\ Theses}
\begin{itemize}
\item Marcel Reif.  Automatische Erkennung von exakten und
  Near-Duplikaten in einer Netzwerkdatenbank, \emph{Bachelor of
    Science}, University of Koblenz--Landau, 2016.
\item Nils Geilen.  Entwicklung eines Systems zur Vorhersage von
  Nutzeraktivität auf den Diskussionsseiten der Wikipedia,
  \emph{Bachelor of Science}, University of Koblenz--Landau, 2015.
\item Jesús Cabello González.  Berechnung und Approximation von
  Kürzeste-Pfad-Statistiken in großen Netzwerken für KONECT,
  \emph{Bachelor of Science}, University of Cádiz, 2014. 
\item Julia Preusse. Analysis of the WebUni Online Student Community,
  \emph{Dipl.-Inform.}, Otto-von-Guericke University Magdeburg, 2010.  
\item Stephan Spiegel.  A Hybrid Approach to Recommender Systems based
  on Matrix Factorization, \emph{Dipl.-Inform.}, Technical University of
  Berlin, 2009.  
\item Christian Banik.  Recommending Wiki Articles using Collaborative
  Filtering, \emph{Dipl.-Inform.}, Technical University of Berlin, 2009. 
\item Iris Breddin.  Untersuchung der Klassifizierbarkeit und
  Klassifikation von Schweißnahtdaten, \emph{Dipl.-Inform.}, Technical
  University of Berlin, 2008.    
\end{itemize}

\section{Conference Organization}
\begin{itemize}
\item \href{https://sites.google.com/site/ocm2016/}{WeST Off-Campus
  Meeting} (OCM), 2016, Chair. 
\item \href{http://informatik2013.de/}{INFORMATIK}, 2013, Publicity Chair. 
\item \href{http://www.websci11.org/}{Web Science Conf.} (WebSci), 2011,
  Publicity Chair.  
\item \href{http://www.dai-labor.de/unm2010/}{Special Session on
  Uncertainty in Network Mining} (UNM) at the Int.\ Conf.\ on Information
  Processing and Management of Uncertainty in Knowledge-based Systems (IPMU),
  2010, Technical Chair.   
\end{itemize}

\section{Program Committees}
\begin{itemize}
\item \href{http://www.icwsm.org/2016/index.php}{Int.\ AAAI Conf.\ on
  Web and Social Media} (ICWSM), 2016.
\item \href{http://www.snow-workshop.org/}{Workshop on Social News on
  the Web} (SNOW) at the World Wide Web Conf.\ (WWW), 2016. 
\item \href{http://events.kmi.open.ac.uk/salsa2016/}{Workshop on SociAL
  Semantic Analysis} (SALSA), 2016.
\item \href{https://failworkshops.wordpress.com/fail-at-ir16/}{\#FAIL!
  -- The Workshop Series} at the Internet Research Conf.\ (IR), 2015.
\item \href{http://www.www2015.it/call-for-web-science-track/}{Web
  Science Track}, Int.\ World Wide Web Conf.\ (WWW), 2015.
\item \href{https://failworkshops.wordpress.com/fail-workshop-at-websci15/}{\#FAIL! -- The Workshop Series} at the Web Science Conf.\ (WebSci), 2015.
\item \href{http://www.websci14.org/}{Web Science Conf.} (WebSci),
  2014. 
\item \href{http://www.snow-workshop.org/}{Workshop on Social News on
  the Web} (SNOW) at the World Wide Web Conf.\ (WWW), 2014.
\item \href{http://www.cool2014.com/}{Workshop on Connecting Online \&
  Offline Life} (COOL) at the World Wide Web Conf.\ (WWW), 2014. 
\item \href{http://www2014.kr/calls/call-for-web-science-track/}{Web
  Science Track}, Int.\ World Wide Web Conf.\ (WWW), 2014.
\item \href{http://ijcai13.org/}{Int.\ Joint Conf.\ on Artificial
  Intelligence} (IJCAI), 2013.  
\item \href{http://www.websci13.org/}{Web Science Conf.} (WebSci), 2013. 
\item \href{http://webscience-education-workshop.blogs.usj.edu.lb/}{Web
  Science Education Workshop} at the Web Science Conf.\ (WebSci), 2013. 
\item \href{http://www.ftsm.ukm.my/stair13/}{Conf.\ on Semantic
  Technology and Information Retrieval} (STAIR), 2013. 
\item \href{http://spim-workshop.org/}{Int.\ Workshop on Semantic
  Personalized Information Management} (SPIM) at the Int.\ Conf.\ on Web
  Search and Data Mining (WSDM), 2013.  
\item \href{http://ecir2013.org/}{European Conf.\ on Information
  Retrieval} (ECIR), 2013.  
\item \href{http://mama.west.uni-koblenz.de/}{Workshop on Metrics,
  Analysis and Tools for Online Community Management} (MAMA), at
  INFORMATIK, 2013
\item \href{http://www.websci12.org/}{Web Science Conf.} (WebSci), 2012. 
\item \href{http://ecir2012.upf.edu/}{European Conf.\ on Information
  Retrieval} (ECIR), poster track, 2012.  
\item \href{http://www.oegai.at/konvens2012/}{Conf.\ on Natural Language
  Processing} (KONVENS), 2012. 
\item \href{http://spim-workshop.org/}{Workshop on Personalized
  Information Management: Linking Social and Semantic Web} (SPIM) at the
  Int.\ Conf.\ on Web Engineering (ICWE), 2012. 
\item \href{http://www.cikm2011.org/}{Conf.\ on Information and Knowledge
  Management} (CIKM), 2011.
\item \href{http://www.websci11.org/}{Web Science Conf.} (WebSci), 2011.
\item \href{http://www.sigkdd.org/kdd/2011/}{Int.\ Conf.\ on 
  Knowledge Discovery and Data Mining} (KDD), research track, 2011.
\item \href{http://www.dai-labor.de/camra2010/}{Challenge on
  Context-aware Movie Recommendation} (CAMRa) at the Conf.\ on  
  Recommender Systems (RecSys), 2010.   
\item \href{http://www.dai-labor.de/unm2010/}{Special Session on Uncertainty in Network Mining} (UNM) at the
  Int.\ Conf.\ on Information Processing and Management of
  Uncertainty in Knowledge-based Systems (IPMU), 2010. 
\end{itemize}

%- \section{Review Boards and Other Reviewing}
%- \begin{itemize}
%- \item \href{http://journals.cambridge.org/action/displayJournal?jid=NWS}{Network Science}, 2015, 2016.
%- \item \href{http://surveys.acm.org/}{ACM Computing Surveys}, 2014, 2015. 
%- \item \href{http://www.webscience-journal.net}{The J.\ of Web Science}, 2014. 
%- \item \href{http://toit.acm.org/}{ACM Trans. on Internet
%- Technologies} (TOIT), 2014.  
%- \item \href{http://www.ii.pwr.wroc.pl/~krol/eng_EPP.htm}{Special
%- Issue on ``Propagation Phenomenon in Complex Networks:  Theory and
%- Practice''}, New Generation Computing, 2014. 
%- \item \href{http://iswc2013.semanticweb.org/}{Int.\ Semantic Web Conf.}
%- (ISWC), 2013.  
%- \item \href{http://www.compstat2012.org/}{Int.\ Conf. on Computational
%- Statistics} (COMPSTAT), 2012.  
%- \item
%- \href{http://www.journals.elsevier.com/computer-networks/}{Int.\ J.\ of
%- Computer and Telecommunications Networking} (COMNET), 2012. 
%- \item \href{http://ieee-cis.org/pubs/tnn/}{IEEE Trans. on Neural
%- Networks} (TNN), 2011.   
%- \item
%- \href{http://www.springer.com/computer/ai/journal/10458}{J.\ on Autonomous
%- Agents and Multi-agent Systems} (AAMAS), Special Issue on Agent Mining,
%- 2011. 
%- \item \href{http://tkdd.cs.uiuc.edu/}{ACM Trans. on Knowledge
%- Discovery from Data} (TKDD), 2011.  
%- \item \href{http://mklab.iti.gr/tvca/}{TV Content Analysis} (TVCA), 2011.
%- \item \href{http://www.loa-cnr.it/ontolex2010}{Workshop on Ontologies
%- and Lexical Resources} (OntoLex) at the Int.\ Conf.\ on Computational
%- Linguistics (COLING), 2010. 
%- \end{itemize}

\newcommand{\bibname}{
%- \hide{
Ten Selected
%- }
Publications
}
%\bibliographystyle{alpha}
%\bibliographystyle{unsrturl}
% The style is defined in publications.tex, not here
\bibliography{kunegis}

\section{Keynotes and Invited Talks}
\begin{itemize}
\item Algebraic Graph-theoric Measures of Conflict, 
  \href{http://jgss.sciencesconf.org/}{Journée Graphes et Systèmes
    Sociaux} (Seminar on Graphs and Social Systems), 2016.  
\item Measuring Conflict in Signed Social Networks, 
  Application of Network Theory on Computational Social Science
  (Workshop), 2015.
\item Large Network Collections:  The Power of Many Datasets,
  \href{http://www.ars15.unisa.it/}{Int.\ Workshop on Social
    Network Analysis} (ARS), 2015. 
\item Network Analysis Tools for Online Communities: The Koblenz Network
  Collection. Keynote, \href{http://mama.west.uni-koblenz.de/}{Workshop
    on Metrics, Analysis and Tools for Online Community Management}
  (MAMA), 2013.  
\end{itemize}

%- \section{Other Talks}
%- \begin{itemize}
%-   \item Generating Networks with Realistic Properties Based on a
%-     Given (Set of) Network(s), and a Short Overview of the KONECT
%-     Project.  Université de Namur, 2016. 
%-   \item Web Science in Practice:  Web Observatories.  WSTNet Web
%-     Science Summer School, 2016.  
%-   \item KONECT: The Koblenz Network Collection -- Towards a Broad
%-     Analysis of Complex Systems.  ETH Zürich, 2015.
%-   \item Observing the Web: The Koblenz Network Collection.
%-     Bournemouth University, 2013.
%-   \item Growth and Decay: Using Machine Learning to Predict the Web's
%-     Future, Workshop on Artificial Intelligence on the Web, 2012. 
%-   \item Models of Like, Dislike, Similarity and Dissimilarity using
%-     Split-complex Numbers, University College Dublin, 2012. 
%-   \item Why Is Beyoncé More Popular Than Me:  Fairness, Diversity and
%-     Other Measures of Network Equality. University of Freiburg, 2012. 
%-   \item Fairness on the Web: Alternatives to the Power
%-     Law. 
%-     Leibniz-Institut für Sozialwissenschaften, Cologne, 2012.  
%-   \item On the Spectral Evolution of Large Networks.  University College
%-     Cork, 2011.   
%-   \item On the Spectral Evolution of Large Networks.  Fraunhofer IAIS,
%-     St.\ Augustin, 2011.  
%-   \item On the Spectral Evolution of Large Networks. Technical
%-     University of Berlin, 2011.  
%-   \item The Slashdot Zoo: Mining a Social Network with Negative
%-     Edges. Data and Knowledge Engineering Research Colloquium,
%-     Otto-von-Guericke University Magdeburg, 2010.  
%-   \item The Slashdot Zoo: Mining a Social Network with Negative
%-     Edges. University of Koblenz--Landau, 2010. 
%-   \item The Slashdot Zoo: Mining a Social Network with Negative
%-     Edges. University of Hannover, 2010. 
%-   \item PIA+COMM~-- an Intelligent Search Engine.
%-     Michael Hahne, Corinna Jung, Jérôme Kunegis, Andreas Lommatzsch and
%-     André Paus. 
%-     Workshop on the Formation of Social Networks in Social Software
%-     Applications, INFORMATIK, 2006. 
%- \end{itemize}

%- \section{Posters without Publication}
%- \begin{itemize}
%-   \item Linguistic Neighbourhoods:  Explaining Cultural Borders on
%-     Wikipedia through Multilingual Co-editing Activity.  Anna
%-     Samoilenko, Fariba Karimi, Daniel Edler, Jérôme Kunegis, Markus
%-     Strohmaier, Int.\ School and Conf.\ on Network Science (NetSciX),
%-     2016. 
%-   \item
%-     Social Network Observatory.  Jérôme Kunegis, Markus Strohmaier,
%-     Steffen Staab.  Computational Social Science Winter Symposium, 2015.
%-   \item 
%-     Quantifying Cultural Similarity through Language Co-occurrences
%-     in Wikipedia Editing Activity. Anna Samoilenko, Fariba Karimi,
%-     Daniel Edler, Jérôme Kunegis, Markus Strohmaier.  Computational
%-     Social Science Winter Symposium, 2015.
%-   \item 
%-     Polarisation in Voting Platforms: A Case Study of LiquidFeedback
%-     in the German Pirate Party. Manuel Mittler, Christoph Carl Kling,
%-     Jérôme Kunegis, Markus Strohmaier, Computational Social Science
%-     Winter Symposium, 2015.
%-   \item
%-     Twitter as a Political Network -- Predicting the Following and
%-     Unfollowing Behavior of {German} Politicians. Julia Perl, Claudia
%-     Wagner, Jérôme Kunegis, Steffen Staab, Web Science Conf., 2015. 
%-   \item 
%-     Online Delegative Democracy: A Case Study of the German Pirate
%-     Party's Voting Platform.  Christoph Carl Kling, Jérôme Kunegis,
%-     Heinrich Hartmann.  Computational Social Science Winter
%-     Symposium, 2014.
%-   \item
%-     Social Networking By Proxy: A Case Study of Catster, Dogster and
%-     Hamsterster. Computational Social Science Winter
%-     Symposium, 2014.
%-   \item 
%-     A Theory-Driven Approach for Link and Unlink Predictions in
%-     Directed Social Networks. Julia Perl, Claudia Wagner, Jérôme
%-     Kunegis, Steffen Staab.  Computational Social Science Winter
%-     Symposium, 2014.
%-   \item 
%-     \href{http://userpages.uni-koblenz.de/~kunegis/paper/kunegis-konect.poster.netsci2012.pdf}{KONECT -- The Koblenz Network Collection}.  
%-     Jérôme Kunegis, Steffen Staab, Daniel Dünker.
%-     Int.\ School and Conf.\ on Network Science, 2012.  
%-   \item
%-     \href{http://uni-koblenz.de/~kunegis/paper/kunegis-konect.poster.essir.pdf}{Need
%-     Networks? KONECT -- The Koblenz Network Collection}.
%-     European Summer School on Information Retrieval, 2011. 
%-   \item
%-     \href{http://uni-koblenz.de/~kunegis/paper/kunegis-konect.poster.pdf}{KONECT~--
%-     The Koblenz Network Collection}. Web Science Conf., 2011. 
%-   \item
%-     \href{http://uni-koblenz.de/~kunegis/paper/kunegis-spread-modes.poster.pdf}{Uncovering
%-     Multi-modal Spread Modes using Joint 
%-     Diagonalization in Dynamic Human Contact Networks}. 
%-     Damien Fay, Jérôme Kunegis, Eiko Yoneki.  Interdisciplinary
%-     Workshop on Information and Decision in Social Networks, 2011. 
%-   \item
%- \href{http://uni-koblenz.de/~kunegis/paper/kunegis-phdthesis.poster.pdf}{On
%- the Spectral Evolution of Large Networks}. Postersession 
%-         für Nachwuchswissenschaftler/innen, University of Koblenz--Landau,
%-         2011. 
%-   \item
%- \href{http://uni-koblenz.de/~kunegis/paper/kunegis-phdthesis.poster-sdm.pdf}{On
%- the Spectral Evolution of Large Networks}. SIAM 
%-     Conf.\ on Data Mining Doctoral Forum, 2010. 
%- \end{itemize}

%- \section{Demonstrations and Presentations}
%- \begin{itemize}
%-  \item Institute for Web Science and Technologies, conference booth.
%-        \href{http://informatik2013.de}{INFORMATIK 2013}.  
%-  \item Schülerinformationstage, Institute for Web Science and
%-        Technologies, University of Koblenz--Landau, 2011, 2013, 2015, 2016.  
%-  \item LiveTweet: Monitoring and Predicting Interesting Microblog
%-        Posts, \href{http://ecir2012.upf.edu/}{European Conference on Information Retrieval} (ECIR),
%-        2012. 
%-  \item Time-aware Centrality in Contact Network Analysis.
%-        Eiko Yoneki, Damien Fay, Jérôme Kunegis.
%-        \href{http://www.eccs2011.eu/}{European Conference on Complex Systems} (ECCS), 2011. 
%-  \item PIA, Spree, CeBIT, 2010.
%-  \item PIA+COMM, CeBIT, 2007.
%-  \item PIA, Lange Nacht der Wissenschaften Berlin/Potsdam, 2006. 
%- \end{itemize}

\section{Mentions in Media}
\begin{itemize}
  \item
    \href{https://www.piratenpartei.de/2015/05/31/die-abstimmungssoftware-liquidfeedback-der-piratenpartei-wegweisend-fuer-demokratie-2-0/}{Die
      Abstimmungssoftware LiquidFeedback in der Piratenpartei:
      wegweisend für Demokratie 2.0?}, www.piratenpartei.de, May 2015. 
\item
  \href{http://www.wired.it/attualita/2015/04/01/come-funzionato-davvero-democrazia-liquida-dei-pirati-tedeschi/}{Come
    ha funzionato davvero la democrazia liquida dei Pirati tedeschi},
  Wired Italia, April 2015. 
\end{itemize}

\section{Awards and Nominations}
\begin{itemize}
  \item First Prize, Int.\ Conf.\ on Web and Social Media (ICWSM) Science Slam, 2016.
  \item Best Poster Award, Int.\ School and Conf.\ on Network Science
    (NetSciX), 2016. 
  \item Best Paper Honorable Mention, Int.\ Conf.\ on Web and Social
    Media (ICWSM), 2015.  
  \item Ted Nelson Newcomer Award Nominee, Conf.\ on Hypertext and
    Social Media (HT), 2012.   
  \item Student Travel Award, Int.\ Conf.\ on Information and Knowledge
    Management (CIKM), 2010. 
  \item Student Travel Fellowship Award, SIAM Conf.\ on Data Mining (SDM),
    2010.  
  \item Best Paper Nominee, Industrial Conf.\ on Data Mining (IndCDM), 2007.    
  \item Premier Prix (First Prize), Rallye mathématique d'Alsace, Terminale (Alsace Mathematical Rally), 1999. 
\end{itemize}

\section{Main Software Authorship}
\begin{itemize}
  \item \href{https://github.com/kunegis/stu}{Stu}, build automation (C++11)
  \item SynGraphy, graph generator (Matlab)
  \item \href{http://konect.uni-koblenz.de/toolbox}{KONECT Toolbox}, network analysis toolbox (Matlab, C99)
  \item \href{https://west.uni-koblenz.de/Research/systems/polcovar}{Polcovar}, software for counting patterns in random graphs (Matlab)
  \item Universal Recommender, recommendation library (Java)
  \item BabyChess, chess engine (C++98)
  \item \emph{More:  see {\tt \href{https://github.com/kunegis}{github.com/kunegis}}}
\end{itemize}

\section{Miscellaneous}
\begin{itemize}
  \item Birthday:  March 27
  \item Languages: French (first language), German (first language),
    English (advanced)
  \item Programming languages and technologies I use extensively:
    Bourne shell, C, C++, Java, Latex, Make, Matlab, sed
%-  \item Other programming languages and technologies I have used:
%-    BASIC, HTML, JavaScript, Logo, Perl, PHP, Turbo Pascal 
\end{itemize}

\end{resume}
\end{document}
