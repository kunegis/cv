\documentclass[line,mm]{res}

\usepackage[utf8]{inputenc}
\usepackage[pdftex]{hyperref}
\usepackage[usenames,dvipsnames]{color}
\usepackage{marginnote}
\usepackage{fontawesome} % Installed by doing this:  https://blog.stefanproell.at/2016/09/14/use-fontawesome-with-pdflatex/
\usepackage{enumitem}
\usepackage{ifthen} % For extbib.tex

\addtolength{\oddsidemargin}{-0.5in}
\addtolength{\evensidemargin}{-0.5in}
\addtolength{\marginparsep}{-0.8in}
\addtolength{\marginparwidth}{+1in}

\setlist[itemize]{topsep=-5pt,itemsep=-1pt}

% Auto-incrementing numbers for each type of publication
\newcounter{x}
\setcounter{x}{1}
\newcommand{\publicationnumber}{\arabic{x}\stepcounter{x}}
\newcounter{y}
\setcounter{y}{1}
\newcommand{\talknumber}{T\arabic{y}\stepcounter{y}}
\newcounter{z}
\setcounter{z}{1}
\newcommand{\popularizationnumber}{\roman{z}\stepcounter{z}}

% AUTOMATICALLY GENERATED by ./mkext on Mon Jul 17 15:07:40 CEST 2017 -- DON'T EDIT
%
% Citations of Jérôme KUNEGIS available through the convenient
% \citek{...} command, where '...' refers to the Bibtex ID of one of
% Jérôme's papers, as used in the file 'kunegis.bib'. 
%
% This does not use Bibtex, but instead will use the number that is used
% in Jérôme's CV.  As a result, you can cite Jérôme's papers and
% conveniently attach his CV to your publication.  Mostly used for
% project proposals, research statements, and similar texts by Jérôme
% himself. 
%
% This only works for publications that have a numerical ID in the CV.
% Other types of publications such as the PhD thesis and posters are not
% supported. 
%
% HOW TO USE
%
% Step (1):  put this file ('extbib.tex') in your directory
% Step (2):  write \usepackage{ifthen} in your Latex file
% Setp (3):  write % AUTOMATICALLY GENERATED by ./mkext on Mon Jul 17 15:07:40 CEST 2017 -- DON'T EDIT
%
% Citations of Jérôme KUNEGIS available through the convenient
% \citek{...} command, where '...' refers to the Bibtex ID of one of
% Jérôme's papers, as used in the file 'kunegis.bib'. 
%
% This does not use Bibtex, but instead will use the number that is used
% in Jérôme's CV.  As a result, you can cite Jérôme's papers and
% conveniently attach his CV to your publication.  Mostly used for
% project proposals, research statements, and similar texts by Jérôme
% himself. 
%
% This only works for publications that have a numerical ID in the CV.
% Other types of publications such as the PhD thesis and posters are not
% supported. 
%
% HOW TO USE
%
% Step (1):  put this file ('extbib.tex') in your directory
% Step (2):  write \usepackage{ifthen} in your Latex file
% Setp (3):  write % AUTOMATICALLY GENERATED by ./mkext on Mon Jul 17 15:07:40 CEST 2017 -- DON'T EDIT
%
% Citations of Jérôme KUNEGIS available through the convenient
% \citek{...} command, where '...' refers to the Bibtex ID of one of
% Jérôme's papers, as used in the file 'kunegis.bib'. 
%
% This does not use Bibtex, but instead will use the number that is used
% in Jérôme's CV.  As a result, you can cite Jérôme's papers and
% conveniently attach his CV to your publication.  Mostly used for
% project proposals, research statements, and similar texts by Jérôme
% himself. 
%
% This only works for publications that have a numerical ID in the CV.
% Other types of publications such as the PhD thesis and posters are not
% supported. 
%
% HOW TO USE
%
% Step (1):  put this file ('extbib.tex') in your directory
% Step (2):  write \usepackage{ifthen} in your Latex file
% Setp (3):  write \input{extbib} in your Latex file
% Step (4):  use a command such as \citek{kunegis:slashdot-zoo} to cite
%            a paper of Jérôme. 
% Step (5):  attach Jérôme's CV ('cv-kunegis.pdf') to your publication  
%

\newcommand{\citek}[1]{\ifthenelse{\equal{#1}{kunegis:nopref}}{[1]}{}\ifthenelse{\equal{#1}{toedtli:ctqw}}{[2]}{}\ifthenelse{\equal{#1}{samoilenko:language-hierarchy}}{[3]}{}\ifthenelse{\equal{#1}{komepol:dud}}{[4]}{}\ifthenelse{\equal{#1}{kunegis:bipartivity}}{[5]}{}\ifthenelse{\equal{#1}{kunegis:spectral-evolution}}{[6]}{}\ifthenelse{\equal{#1}{fay:appcluster}}{[7]}{}\ifthenelse{\equal{#1}{hahne:gradualisiertes-communitymodell}}{[8]}{}\ifthenelse{\equal{#1}{schelter:nzh-paper}}{[9]}{}\ifthenelse{\equal{#1}{schelter:trackers}}{[10]}{}\ifthenelse{\equal{#1}{homscheid:lkml}}{[11]}{}\ifthenelse{\equal{#1}{kling:lqfb}}{[12]}{}\ifthenelse{\equal{#1}{kling:mgtm}}{[13]}{}\ifthenelse{\equal{#1}{kunegis:preferential-attachment}}{[14]}{}\ifthenelse{\equal{#1}{kunegis:latent-negative}}{[15]}{}\ifthenelse{\equal{#1}{fay:joint-diagonalization}}{[16]}{}\ifthenelse{\equal{#1}{preusse:unlink-prediction}}{[17]}{}\ifthenelse{\equal{#1}{kunegis:directed-decomposition}}{[18]}{}\ifthenelse{\equal{#1}{kunegis:power-law}}{[19]}{}\ifthenelse{\equal{#1}{kunegis:network-rank}}{[20]}{}\ifthenelse{\equal{#1}{nasir:retweet-ranking}}{[21]}{}\ifthenelse{\equal{#1}{nasir:retweets}}{[22]}{}\ifthenelse{\equal{#1}{kunegis:spectral-network-evolution}}{[23]}{}\ifthenelse{\equal{#1}{kunegis:hyperbolic-sine}}{[24]}{}\ifthenelse{\equal{#1}{kunegis:signed-kernels}}{[25]}{}\ifthenelse{\equal{#1}{karatas:interactive-tv}}{[26]}{}\ifthenelse{\equal{#1}{kunegis:spectral-transformation}}{[27]}{}\ifthenelse{\equal{#1}{kunegis:slashdot-zoo}}{[28]}{}\ifthenelse{\equal{#1}{kunegis:alternative-similarity}}{[29]}{}\ifthenelse{\equal{#1}{kunegis:netflix-srd}}{[30]}{}\ifthenelse{\equal{#1}{kunegis:assessing-unrated}}{[31]}{}\ifthenelse{\equal{#1}{kunegis:adapting-ratings}}{[32]}{}\ifthenelse{\equal{#1}{kunegis:negative-resistance}}{[33]}{}\ifthenelse{\equal{#1}{lommatzsch:multi-agent}}{[34]}{}\ifthenelse{\equal{#1}{mehlitz:ir-measure}}{[35]}{}\ifthenelse{\equal{#1}{schmidt:bandwidth-optimization}}{[36]}{}\ifthenelse{\equal{#1}{mehlitz:evaluation-measure}}{[37]}{}\ifthenelse{\equal{#1}{sun:wiki-talk}}{[38]}{}\ifthenelse{\equal{#1}{kunegis:konect}}{[39]}{}\ifthenelse{\equal{#1}{kunegis:split-complex-dating}}{[40]}{}\ifthenelse{\equal{#1}{kunegis:konect-cloud}}{[41]}{}\ifthenelse{\equal{#1}{spiegel:time-tensor}}{[42]}{}\ifthenelse{\equal{#1}{deluca:multilingual}}{[43]}{}\ifthenelse{\equal{#1}{spiegel:hydra}}{[44]}{}\ifthenelse{\equal{#1}{kunegis:kernel-scalability}}{[45]}{}\ifthenelse{\equal{#1}{milosevic:update-policy}}{[46]}{}\ifthenelse{\equal{#1}{samoilenko:neighbourhoods-ic2s2-2017}}{[47]}{}\ifthenelse{\equal{#1}{schelter:nzh-ea}}{[48]}{}\ifthenelse{\equal{#1}{kunegis:nopref-netsci}}{[49]}{}\ifthenelse{\equal{#1}{duenker:petster}}{[50]}{}\ifthenelse{\equal{#1}{perl:twitter-unlink}}{[51]}{}\ifthenelse{\equal{#1}{samoilenko:wikilang}}{[52]}{}\ifthenelse{\equal{#1}{kunegis:konect-netsci}}{[53]}{}\ifthenelse{\equal{#1}{che-alhadi:live-tweet}}{[54]}{}\ifthenelse{\equal{#1}{yoneki:time-aware-centrality}}{[55]}{}\ifthenelse{\equal{#1}{che-alhadi:trec-microblog}}{[56]}{}\ifthenelse{\equal{#1}{gottron:rule}}{[57]}{}\ifthenelse{\equal{#1}{gottron:challenges}}{[58]}{}\ifthenelse{\equal{#1}{said:exploiting-hierarchical-tags}}{[59]}{}\ifthenelse{\equal{#1}{kunegis:network-encyclopedia-2}}{[60]}{}\ifthenelse{\equal{#1}{kunegis:network-encyclopedia-1}}{[61]}{}\ifthenelse{\equal{#1}{kunegis:konect-handbook}}{[62]}{}\ifthenelse{\equal{#1}{schelter:trackers-journal}}{[63]}{}\ifthenelse{\equal{#1}{kunegis:petster}}{[64]}{}\ifthenelse{\equal{#1}{kunegis:negativity}}{[65]}{}\ifthenelse{\equal{#1}{perl:decline}}{[66]}{}\ifthenelse{\equal{#1}{kunegis:polcovar}}{[67]}{}\ifthenelse{\equal{#1}{fay:centrality}}{[68]}{}\ifthenelse{\equal{#1}{kunegis:universal-recommender}}{[69]}{}\ifthenelse{\equal{#1}{kunegis:fluorine-compounds}}{[70]}{}\ifthenelse{\equal{#1}{kunegis:german-vowels}}{[71]}{}\ifthenelse{\equal{#1}{kunegis:english-vowels}}{[72]}{}\ifthenelse{\equal{#1}{kunegis:matrix-taxonomy}}{[73]}{}}
 in your Latex file
% Step (4):  use a command such as \citek{kunegis:slashdot-zoo} to cite
%            a paper of Jérôme. 
% Step (5):  attach Jérôme's CV ('cv-kunegis.pdf') to your publication  
%

\newcommand{\citek}[1]{\ifthenelse{\equal{#1}{kunegis:nopref}}{[1]}{}\ifthenelse{\equal{#1}{toedtli:ctqw}}{[2]}{}\ifthenelse{\equal{#1}{samoilenko:language-hierarchy}}{[3]}{}\ifthenelse{\equal{#1}{komepol:dud}}{[4]}{}\ifthenelse{\equal{#1}{kunegis:bipartivity}}{[5]}{}\ifthenelse{\equal{#1}{kunegis:spectral-evolution}}{[6]}{}\ifthenelse{\equal{#1}{fay:appcluster}}{[7]}{}\ifthenelse{\equal{#1}{hahne:gradualisiertes-communitymodell}}{[8]}{}\ifthenelse{\equal{#1}{schelter:nzh-paper}}{[9]}{}\ifthenelse{\equal{#1}{schelter:trackers}}{[10]}{}\ifthenelse{\equal{#1}{homscheid:lkml}}{[11]}{}\ifthenelse{\equal{#1}{kling:lqfb}}{[12]}{}\ifthenelse{\equal{#1}{kling:mgtm}}{[13]}{}\ifthenelse{\equal{#1}{kunegis:preferential-attachment}}{[14]}{}\ifthenelse{\equal{#1}{kunegis:latent-negative}}{[15]}{}\ifthenelse{\equal{#1}{fay:joint-diagonalization}}{[16]}{}\ifthenelse{\equal{#1}{preusse:unlink-prediction}}{[17]}{}\ifthenelse{\equal{#1}{kunegis:directed-decomposition}}{[18]}{}\ifthenelse{\equal{#1}{kunegis:power-law}}{[19]}{}\ifthenelse{\equal{#1}{kunegis:network-rank}}{[20]}{}\ifthenelse{\equal{#1}{nasir:retweet-ranking}}{[21]}{}\ifthenelse{\equal{#1}{nasir:retweets}}{[22]}{}\ifthenelse{\equal{#1}{kunegis:spectral-network-evolution}}{[23]}{}\ifthenelse{\equal{#1}{kunegis:hyperbolic-sine}}{[24]}{}\ifthenelse{\equal{#1}{kunegis:signed-kernels}}{[25]}{}\ifthenelse{\equal{#1}{karatas:interactive-tv}}{[26]}{}\ifthenelse{\equal{#1}{kunegis:spectral-transformation}}{[27]}{}\ifthenelse{\equal{#1}{kunegis:slashdot-zoo}}{[28]}{}\ifthenelse{\equal{#1}{kunegis:alternative-similarity}}{[29]}{}\ifthenelse{\equal{#1}{kunegis:netflix-srd}}{[30]}{}\ifthenelse{\equal{#1}{kunegis:assessing-unrated}}{[31]}{}\ifthenelse{\equal{#1}{kunegis:adapting-ratings}}{[32]}{}\ifthenelse{\equal{#1}{kunegis:negative-resistance}}{[33]}{}\ifthenelse{\equal{#1}{lommatzsch:multi-agent}}{[34]}{}\ifthenelse{\equal{#1}{mehlitz:ir-measure}}{[35]}{}\ifthenelse{\equal{#1}{schmidt:bandwidth-optimization}}{[36]}{}\ifthenelse{\equal{#1}{mehlitz:evaluation-measure}}{[37]}{}\ifthenelse{\equal{#1}{sun:wiki-talk}}{[38]}{}\ifthenelse{\equal{#1}{kunegis:konect}}{[39]}{}\ifthenelse{\equal{#1}{kunegis:split-complex-dating}}{[40]}{}\ifthenelse{\equal{#1}{kunegis:konect-cloud}}{[41]}{}\ifthenelse{\equal{#1}{spiegel:time-tensor}}{[42]}{}\ifthenelse{\equal{#1}{deluca:multilingual}}{[43]}{}\ifthenelse{\equal{#1}{spiegel:hydra}}{[44]}{}\ifthenelse{\equal{#1}{kunegis:kernel-scalability}}{[45]}{}\ifthenelse{\equal{#1}{milosevic:update-policy}}{[46]}{}\ifthenelse{\equal{#1}{samoilenko:neighbourhoods-ic2s2-2017}}{[47]}{}\ifthenelse{\equal{#1}{schelter:nzh-ea}}{[48]}{}\ifthenelse{\equal{#1}{kunegis:nopref-netsci}}{[49]}{}\ifthenelse{\equal{#1}{duenker:petster}}{[50]}{}\ifthenelse{\equal{#1}{perl:twitter-unlink}}{[51]}{}\ifthenelse{\equal{#1}{samoilenko:wikilang}}{[52]}{}\ifthenelse{\equal{#1}{kunegis:konect-netsci}}{[53]}{}\ifthenelse{\equal{#1}{che-alhadi:live-tweet}}{[54]}{}\ifthenelse{\equal{#1}{yoneki:time-aware-centrality}}{[55]}{}\ifthenelse{\equal{#1}{che-alhadi:trec-microblog}}{[56]}{}\ifthenelse{\equal{#1}{gottron:rule}}{[57]}{}\ifthenelse{\equal{#1}{gottron:challenges}}{[58]}{}\ifthenelse{\equal{#1}{said:exploiting-hierarchical-tags}}{[59]}{}\ifthenelse{\equal{#1}{kunegis:network-encyclopedia-2}}{[60]}{}\ifthenelse{\equal{#1}{kunegis:network-encyclopedia-1}}{[61]}{}\ifthenelse{\equal{#1}{kunegis:konect-handbook}}{[62]}{}\ifthenelse{\equal{#1}{schelter:trackers-journal}}{[63]}{}\ifthenelse{\equal{#1}{kunegis:petster}}{[64]}{}\ifthenelse{\equal{#1}{kunegis:negativity}}{[65]}{}\ifthenelse{\equal{#1}{perl:decline}}{[66]}{}\ifthenelse{\equal{#1}{kunegis:polcovar}}{[67]}{}\ifthenelse{\equal{#1}{fay:centrality}}{[68]}{}\ifthenelse{\equal{#1}{kunegis:universal-recommender}}{[69]}{}\ifthenelse{\equal{#1}{kunegis:fluorine-compounds}}{[70]}{}\ifthenelse{\equal{#1}{kunegis:german-vowels}}{[71]}{}\ifthenelse{\equal{#1}{kunegis:english-vowels}}{[72]}{}\ifthenelse{\equal{#1}{kunegis:matrix-taxonomy}}{[73]}{}}
 in your Latex file
% Step (4):  use a command such as \citek{kunegis:slashdot-zoo} to cite
%            a paper of Jérôme. 
% Step (5):  attach Jérôme's CV ('cv-kunegis.pdf') to your publication  
%

\newcommand{\citek}[1]{\ifthenelse{\equal{#1}{kunegis:nopref}}{[1]}{}\ifthenelse{\equal{#1}{toedtli:ctqw}}{[2]}{}\ifthenelse{\equal{#1}{samoilenko:language-hierarchy}}{[3]}{}\ifthenelse{\equal{#1}{komepol:dud}}{[4]}{}\ifthenelse{\equal{#1}{kunegis:bipartivity}}{[5]}{}\ifthenelse{\equal{#1}{kunegis:spectral-evolution}}{[6]}{}\ifthenelse{\equal{#1}{fay:appcluster}}{[7]}{}\ifthenelse{\equal{#1}{hahne:gradualisiertes-communitymodell}}{[8]}{}\ifthenelse{\equal{#1}{schelter:nzh-paper}}{[9]}{}\ifthenelse{\equal{#1}{schelter:trackers}}{[10]}{}\ifthenelse{\equal{#1}{homscheid:lkml}}{[11]}{}\ifthenelse{\equal{#1}{kling:lqfb}}{[12]}{}\ifthenelse{\equal{#1}{kling:mgtm}}{[13]}{}\ifthenelse{\equal{#1}{kunegis:preferential-attachment}}{[14]}{}\ifthenelse{\equal{#1}{kunegis:latent-negative}}{[15]}{}\ifthenelse{\equal{#1}{fay:joint-diagonalization}}{[16]}{}\ifthenelse{\equal{#1}{preusse:unlink-prediction}}{[17]}{}\ifthenelse{\equal{#1}{kunegis:directed-decomposition}}{[18]}{}\ifthenelse{\equal{#1}{kunegis:power-law}}{[19]}{}\ifthenelse{\equal{#1}{kunegis:network-rank}}{[20]}{}\ifthenelse{\equal{#1}{nasir:retweet-ranking}}{[21]}{}\ifthenelse{\equal{#1}{nasir:retweets}}{[22]}{}\ifthenelse{\equal{#1}{kunegis:spectral-network-evolution}}{[23]}{}\ifthenelse{\equal{#1}{kunegis:hyperbolic-sine}}{[24]}{}\ifthenelse{\equal{#1}{kunegis:signed-kernels}}{[25]}{}\ifthenelse{\equal{#1}{karatas:interactive-tv}}{[26]}{}\ifthenelse{\equal{#1}{kunegis:spectral-transformation}}{[27]}{}\ifthenelse{\equal{#1}{kunegis:slashdot-zoo}}{[28]}{}\ifthenelse{\equal{#1}{kunegis:alternative-similarity}}{[29]}{}\ifthenelse{\equal{#1}{kunegis:netflix-srd}}{[30]}{}\ifthenelse{\equal{#1}{kunegis:assessing-unrated}}{[31]}{}\ifthenelse{\equal{#1}{kunegis:adapting-ratings}}{[32]}{}\ifthenelse{\equal{#1}{kunegis:negative-resistance}}{[33]}{}\ifthenelse{\equal{#1}{lommatzsch:multi-agent}}{[34]}{}\ifthenelse{\equal{#1}{mehlitz:ir-measure}}{[35]}{}\ifthenelse{\equal{#1}{schmidt:bandwidth-optimization}}{[36]}{}\ifthenelse{\equal{#1}{mehlitz:evaluation-measure}}{[37]}{}\ifthenelse{\equal{#1}{sun:wiki-talk}}{[38]}{}\ifthenelse{\equal{#1}{kunegis:konect}}{[39]}{}\ifthenelse{\equal{#1}{kunegis:split-complex-dating}}{[40]}{}\ifthenelse{\equal{#1}{kunegis:konect-cloud}}{[41]}{}\ifthenelse{\equal{#1}{spiegel:time-tensor}}{[42]}{}\ifthenelse{\equal{#1}{deluca:multilingual}}{[43]}{}\ifthenelse{\equal{#1}{spiegel:hydra}}{[44]}{}\ifthenelse{\equal{#1}{kunegis:kernel-scalability}}{[45]}{}\ifthenelse{\equal{#1}{milosevic:update-policy}}{[46]}{}\ifthenelse{\equal{#1}{samoilenko:neighbourhoods-ic2s2-2017}}{[47]}{}\ifthenelse{\equal{#1}{schelter:nzh-ea}}{[48]}{}\ifthenelse{\equal{#1}{kunegis:nopref-netsci}}{[49]}{}\ifthenelse{\equal{#1}{duenker:petster}}{[50]}{}\ifthenelse{\equal{#1}{perl:twitter-unlink}}{[51]}{}\ifthenelse{\equal{#1}{samoilenko:wikilang}}{[52]}{}\ifthenelse{\equal{#1}{kunegis:konect-netsci}}{[53]}{}\ifthenelse{\equal{#1}{che-alhadi:live-tweet}}{[54]}{}\ifthenelse{\equal{#1}{yoneki:time-aware-centrality}}{[55]}{}\ifthenelse{\equal{#1}{che-alhadi:trec-microblog}}{[56]}{}\ifthenelse{\equal{#1}{gottron:rule}}{[57]}{}\ifthenelse{\equal{#1}{gottron:challenges}}{[58]}{}\ifthenelse{\equal{#1}{said:exploiting-hierarchical-tags}}{[59]}{}\ifthenelse{\equal{#1}{kunegis:network-encyclopedia-2}}{[60]}{}\ifthenelse{\equal{#1}{kunegis:network-encyclopedia-1}}{[61]}{}\ifthenelse{\equal{#1}{kunegis:konect-handbook}}{[62]}{}\ifthenelse{\equal{#1}{schelter:trackers-journal}}{[63]}{}\ifthenelse{\equal{#1}{kunegis:petster}}{[64]}{}\ifthenelse{\equal{#1}{kunegis:negativity}}{[65]}{}\ifthenelse{\equal{#1}{perl:decline}}{[66]}{}\ifthenelse{\equal{#1}{kunegis:polcovar}}{[67]}{}\ifthenelse{\equal{#1}{fay:centrality}}{[68]}{}\ifthenelse{\equal{#1}{kunegis:universal-recommender}}{[69]}{}\ifthenelse{\equal{#1}{kunegis:fluorine-compounds}}{[70]}{}\ifthenelse{\equal{#1}{kunegis:german-vowels}}{[71]}{}\ifthenelse{\equal{#1}{kunegis:english-vowels}}{[72]}{}\ifthenelse{\equal{#1}{kunegis:matrix-taxonomy}}{[73]}{}}


\makeatletter
\makeatother

\definecolor{urlcolor}{rgb}{0.1, 0.1, 0.9}

\hypersetup{ 
  colorlinks=true, pdftitle={Dr. Jérôme KUNEGIS},
  pdfauthor={Jérôme KUNEGIS}, 
  urlcolor=urlcolor
} 

\hyphenation{Ku-ne-gis Schaar-schmidt Chris-toph Web-logs Mag-de-burg KO-NECT}

\begin{document}

\name{
  \hspace{3.95cm} Jérôme KUNEGIS \hfill
  {\normalfont (Complete CV\footnote{This version of this file is up to date as of \input{date}})}
  \hspace{2.95cm}
}
\begin{resume}

\hspace{0.61cm}
\begin{tabular}{lcr}
  {\tt \href{mailto:kunegis@gmail.com}{kunegis@gmail.com}} &\hspace{2.05cm}\hfill &{\tt \href{http://www.linkedin.com/in/kunegis}{linkedin.com/in/kunegis}} \\
  {\tt \href{http://networkscience.wordpress.com/}{networkscience.wordpress.com}} && {\tt \href{https://github.com/kunegis}{github.com/kunegis}} 
\end{tabular}

\vspace{0.1cm}

\section{Education}

\begin{itemize}

\item[{[E1]}]
\marginnote{\href{https://kola.opus.hbz-nrw.de/frontdoor/index/index/docId/581}{\faFileTextO}
  \href{https://github.com/kunegis/pdfs/blob/master/kunegis:phd.poster.pdf}{\faFilePictureO}
  \href{https://www.slideshare.net/kunegis/on-the-spectral-evolution-of-large-networks-phd-thesis-by-jrme-kunegis}{\faSlideshare}
  \href{https://github.com/kunegis/phd}{\faCog}
}
\textbf{Dr.\ rer.\ nat.}, University of Koblenz--Landau, Germany \hfill 2011 \\
(Computer science, PhD-equivalent) \\
Thesis: \emph{\href{https://kola.opus.hbz-nrw.de/frontdoor/index/index/docId/581}{On
       the Spectral Evolution of Large Networks}}

\item[{[E2]}]
\marginnote{\href{https://pdfs.semanticscholar.org/42ca/e9b00864ee09bc9bc92ef3d411e20b966a8d.pdf}{\faFileTextO}}
\textbf{Dipl.-Inform.}, Technical University of Berlin \hfill 2006 \\
(Computer science, Master-equivalent) \\
Thesis: \emph{\href{https://pdfs.semanticscholar.org/42ca/e9b00864ee09bc9bc92ef3d411e20b966a8d.pdf}{Using Integer Linear Programming for Search Results Optimization}}

\item[{[E3]}]
\textbf{Vordiplom}, Technical University of Berlin \hfill 2003 \\
(Computer science, Bachelor-equivalent) \\
Studiengang:  Informatik

\item[{[E4]}]
  \textbf{Baccalauréat}, Lycée Français de Berlin \hfill 1999 \\
  (French high school diploma) \\
Série scientifique

\item[{[E5]}]
\textbf{Abitur}, Lycée Français de Berlin \hfill 1999 \\
  (German high school diploma) \\
Leistungskurse:  Mathematik, Physik 

\item[{[E6]}]
\textbf{Diplôme national du brevet}, Lycée Français de Berlin \hfill 1996 \\
Série collège

\end{itemize}

\section{Positions}
\begin{itemize}
\item[] \textbf{Dev/Systems Engineer} \hfill 2023--present \\
  Society for Worldwide Interbank Financial Telecommunication
\item[] \textbf{Postdoctoral Researcher} \hfill 2017--2019 \\
  Namur Center for Complex Systems (naXys), University of Namur 
\item[] \textbf{Postdoctoral Researcher} \hfill 2011--2016 \\
  Institute for Web Science and Technologies, University of Koblenz--Landau 
\item[] \textbf{Visiting Postdoctoral Researcher} \hfill 2013 \\
  Networks and Operating Systems Group, University of Cambridge
\item[] \textbf{Research Assistant} \hfill 2010--2011 \\
  Institute for Web Science and Technologies, University of Koblenz--Landau 
\item[]   \textbf{Research Assistant} \hfill 2006--2010 \\
  DAI Laboratory, Technical University of Berlin 
\item[] \textbf{Student Research Assistant} \hfill 2002--2006 \\
  DAI Laboratory, Technical University of Berlin
\item[] \textbf{Intern, Air Traffic Control Tower} \hfill 1997 \\
  Berlin-Tempelhof Airport
\end{itemize}

\section{Research Projects}
\begin{itemize}
\item \href{http://konect.cc/}{KONECT} \hfill 2011--present \\
  Koblenz Network Collection
\item \href{http://nouvelles.unamur.be/upnews.2015-10-01.8995593781}{IDEES} (ERDF/FEDER -- Wallonia/Wallonie) \hfill 2017--2019 \\
  L'Internet de Demain pour développer les Entreprises, l'Économie et la Société 
\item \href{http://revealproject.eu/}{REVEAL} (EU FP7-ICT) \hfill 2014--2016 \\
  Revealing Hidden Concepts in Social Media
\item SIP (DFG WGI) \hfill 2014--2016 \\
  Social Information Processing -- Cloud Infrastructure at the Univ.\ of Koblenz--Landau
\item \href{http://www.socialsensor.eu/}{SocialSensor} (EU FP7-ICT) \hfill 2013--2014 \\
  Sensing User Generated Input for Improved Media Discovery and Experience
\item \href{https://books.google.be/books?id=06fDCQAAQBAJ&pg=PA107}{KONECT Cloud} \hfill 2012--2013 \\
  Large Scale Network Mining in the Cloud
\item \href{http://www.robust-project.eu/}{ROBUST} (EU FP7-ICT) \hfill 2010--2013 \\
  Risk and Opportunity Management of Huge-Scale Business Community Cooperation 
\item \href{http://www.weknowit.eu/}{WeKnowIt} (EU FP7-ICT) \hfill 2010 \\
   Emerging, Collective Intelligence for Personal, Organisationaland Social Use
\item MULTIPLA (DFG) \hfill 2010 \\
  Multi-Ontology Learning:  Crossing the Boundaries of Domains and Languages
\item \href{http://www.dai-labor.de/irml/lsr/}{LSR} \hfill 2010 \\
  Learning Semantic Recommenders
\item \href{http://www.dai-labor.de/en/irml/ucpm/}{UCPM} \hfill 2010 \\
  User Centric Profile Management
\item \href{http://www.dai-labor.de/en/irml/serum/}{SERUM} \hfill 2010 \\
  Semantic Recommenders Based on Large Unstructured Datasets
\item \href{http://www.connected-living.org/}{Connected Living} \hfill 2009--2010 \\
   Innovationszentrum ``Vernetztes Leben'' -- Innovation Center ``Connected Living''
\item \href{http://www1.smart-senior.de/}{SmartSenior} \hfill 2009--2010 \\
  Subproject Service Infrastructure and Usability Engineering -- Intelligent Services for Elderly People
\item WebTV \hfill 2009--2010 \\
  Semantic TV Recommendations
\item \href{http://pia-services.de/}{PIA} \hfill 2006--2010 \\
  Personal Information Agent 
\end{itemize}

\section{Lecturing}
\begin{itemize}
\item Graph Theory (Théorie des graphes, in French), Mathematics (BSc),
  Univ.\ of Namur, First Quadrimester 2017/18, 2018/19.   
\item Network Theory and Dynamic Systems (in English), Web Science (MSc), Univ.\ of
  Koblenz--Landau, Summer Term 2014, 2015, 2016.   
\item Introduction to Database Systems (Grundlagen der Datenbanken, in
  German), Computer Science (BSc), Univ.\ of Koblenz--Lan\-dau, 
  Winter Term 2012/\allowbreak 13, 2013/\allowbreak 14, 2014/15, 2015/16.
\end{itemize}

\section{Other Teaching}
\begin{itemize}
\item Advanced Topics in Network Science (Seminar), Univ.\ of
  Koblenz--Landau, Summer Term 2016. 
\item Learning Analytics: Aspects of Machine Learning and Empirical Psychology in E-Learning (Learning
    Analytics: Aspekte des Machine Learnings und empirischer Psychologie
    beim E-Learning, Proseminar), Univ.\ of Koblenz--Landau, Winter Term 2015/2016.
\item Recommender Systems (Seminar), Univ.\ of Koblenz--Landau, Summer Term 2015.
\item Distributed Scalable Network Analysis (Research Lab), Univ.\ of
  Koblenz--Landau, Winter Term 2014/2015. 
\item Advanced Topics in Network Theory (Seminar), Univ.\ of
  Koblenz--Landau, Summer Term 2014. 
\item Network Theory and Dynamic Systems, Web Science (MSc), tutoring, Univ.\ of
  Koblenz--Landau, Summer Term 2013, 2014.
\item Social Networks (Soziale Netzwerke, Proseminar), Univ.\ of
  Koblenz--Landau, Summer Term 2012, 2013.
\item
  Introduction to Database Systems (Grundlagen der Datenbanken), tutoring,
  Computer Science (BSc), Univ.\ of
  Koblenz--Landau, Winter Term 2010/2011, 2011/2012. 
\end{itemize}

\section{Supervised Theses}
\begin{itemize}
\item 
  \marginnote{\href{https://aleph1.uni-koblenz.de/F?func=find-b&find_code=wrd&request=Automatische Erkennung von exakten und Near-Duplikaten in einer Netzwerkdatenbank}{\faFileTextO}}
  Marcel Reif.  Automatische Erkennung von exakten und
  Near-Duplikaten in einer Netzwerkdatenbank, \emph{BSc}, Univ.\ of Koblenz--Landau, 2016.
\item 
  \marginnote{\href{https://aleph1.uni-koblenz.de/F?func=find-b&find_code=wrd&request=Entwicklung eines Systems zur Vorhersage von Nutzeraktivit\%C3\%A4t auf den Diskussionsseiten der Wikipedia}{\faFileTextO}}
  Nils Geilen.  Entwicklung eines Systems zur Vorhersage von
  Nutzeraktivität auf den Diskussionsseiten der Wikipedia,
  \emph{BSc}, Univ.\ of Koblenz--Landau, 2015.
\item
  \marginnote{\href{https://west.uni-koblenz.de/sites/default/files/studying/theses-files/bachelorarbeit_jesus_cabello_gonzalez.pdf}{\faFileTextO}}
  Jesús Cabello González.  Berechnung und Approximation von
  Kürzeste-Pfad-Statistiken in großen Netzwerken für KONECT,
  \emph{BSc}, Univ.\ of Cádiz, 2014. 
\item
  \marginnote{\href{http://fuzzy.cs.ovgu.de/aigaion/index.php/publications/show/800}{\faFileTextO}}
  Julia Preusse. Analysis of the WebUni Online Student Community,
  \emph{Dipl.-Inform.}, Otto von Guericke Univ.\ Magdeburg, 2010.  
\item
  \marginnote{\href{http://www.dai-labor.de/fileadmin/files/publications/DiplomaThesisStephanSpiegel.pdf}{\faFileTextO}}
  Stephan Spiegel.  A Hybrid Approach to Recommender Systems based
  on Matrix Factorization, \emph{Dipl.-Inform.}, Tech.\ Univ.\ Berlin, 2009.  
\item
  \marginnote{\href{https://www.aot.tu-berlin.de/index.php?id=813&dID=190}{\faFileTextO}}
  Christian Banik.  Recommending Wiki Articles using Collaborative
  Filtering, \emph{Dipl.-Inform.}, Tech.\ Univ.\ Berlin, 2009. 
\item
  \marginnote{\href{https://www.aot.tu-berlin.de/index.php?id=813&dID=252}{\faFileTextO}}
  Iris Breddin.  Untersuchung der Klassifizierbarkeit und
  Klassifikation von Schweißnahtdaten, \emph{Dipl.-Inform.}, Tech.\ Univ.\ Berlin, 2008.
\end{itemize}

\section{Conference Organization}
\begin{itemize}
\item
  \marginnote{\href{http://socinf-maison-2018.isistan.unicen.edu.ar/}{\faChain}}
  \href{http://socinf-maison-2018.isistan.unicen.edu.ar/}{Joint International Workshop on Social Influence Analysis and Minining Actionable Insights from Social Networks}\ (SocInf+MAISoN) at the Int.\ Jt.\ Conf.\ on Artif.\ Intell.\ and Eur.\ Conf.\ on Artif.\ Intell.\ (IJCAI-ECAI), 2018, Stockholm, Sweden, Program Chair.
\item 
  \marginnote{\href{https://sites.google.com/site/ocm2016/}{\faChain}}
  \href{https://sites.google.com/site/ocm2016/}{WeST Off-Campus
  Meeting} (OCM), 2016, La~Roche-en-Ardenne, Belgium, Chair. 
\item
  \marginnote{\href{http://informatik2013.de/}{\faChain}}
  \href{http://informatik2013.de/}{INFORMATIK} (largest German computer science conference), 2013, Koblenz, Germany, Publicity Chair. 
\item
  \marginnote{\href{http://www.websci11.org/}{\faChain}}
  \href{http://www.websci11.org/}{Web Science Conference}, 2011, Koblenz, Germany,
  Publicity Chair.  
\item
  \marginnote{\href{http://www.dai-labor.de/unm2010/}{\faChain}}
  \href{http://www.dai-labor.de/unm2010/}{Special Session on
  Uncertainty in Network Mining} (UNM) at the Int.\ Conf.\ on Inf.\ Processing and Manag.\ of Uncertainty in Knowl.-based Syst.\ (IPMU),
  2010, Dortmund, Germany, Technical Chair.   
\end{itemize}

\section{Articles}           \input{list-article}
\section{Conference Papers}  \input{list-conf}
\section{Workshop Papers}    \input{list-workshop}
\section{Reference Works}    \input{list-reference}
\section{Abstracts}          \input{list-abstract}
\section{Other Publications} \input{list-other}

\section{Keynotes and Invited Talks}
\begin{itemize}
\item[{[\talknumber]}]
  \marginnote{\href{http://databeers.brussels/}{\faLink}
  \href{https://www.youtube.com/watch?v=sK69pa3aEjk&feature=youtu.be}{\faVideoCamera}}
  Cats, Dogs, and Hamsters:  The Secret Online Network of Pet Owners,
  \href{http://databeers.brussels/}{DataBeers Brussels}, 2017.   
\item[{[\talknumber]}] 
  \marginnote{\href{http://jgss.sciencesconf.org/}{\faLink}
    \href{https://www.slideshare.net/kunegis/algebraic-graphtheoretic-measures-of-conflict}{\faSlideshare}}
  \href{https://www.slideshare.net/kunegis/algebraic-graphtheoretic-measures-of-conflict}{Algebraic Graph-theoric Measures of Conflict}, 
  \href{http://jgss.sciencesconf.org/}{Journée Graphes et Systèmes
    Sociaux} (Seminar on Graphs and Soc.\ Syst.), 2016.  
\item[{[\talknumber]}] Measuring Conflict in Signed Social Networks, 
  Application of Netw.\ Theory on Comput.\ Soc.\ Sci.\ (Workshop), 2015.
\item[{[\talknumber]}] 
  \marginnote{\href{http://www.ars15.unisa.it/}{\faLink}}
  Large Network Collections:  The Power of Many Datasets,
  \href{http://www.ars15.unisa.it/}{Int.\ Workshop on Soc.\ Netw.\ Anal.}\ (ARS), 2015. 
\item[{[\talknumber]}] 
  \marginnote{\href{http://mama.west.uni-koblenz.de/}{\faLink}
    \href{https://github.com/kunegis/pdfs/blob/master/kunegis:konect-mama.presentation.pdf}{\faSlideshare}}
  \href{https://github.com/kunegis/pdfs/blob/master/kunegis:konect-mama.presentation.pdf}{Network
    Analysis Tools for Online Communities: The Koblenz Network 
    Collection.} Keynote, \href{http://mama.west.uni-koblenz.de/}{Workshop
    on Metrics, Anal.\ and Tools for Online Community Manag.}\ (MAMA), 2013.  
\end{itemize}

\section{Academic Tutorials}
\begin{itemize}
  \item[{[\talknumber]}]
    \marginnote{\href{http://xn.unamur.be/network-collection-tutorial-cikm2017/}{\faLink}
    \href{http://xn.unamur.be/network-collection-tutorial-cikm2017/\#slides}{\faSlideshare}}
    \href{http://xn.unamur.be/network-collection-tutorial-cikm2017/}{Network Analysis in the Age of Large Network Dataset Collections~--
    Challenges, Solutions and Applications.}
    Jérôme Kunegis, Renaud Lambiotte, Int.\ Conf.\ on Inf.\ and Knowl.\ Manag., 2017.  
  \item[{[\talknumber]}] 
    \marginnote{\href{http://wwsss16.webscience.org/speakers-and-tutors/dr-jerome-kunegis}{\faLink}
      \href{https://github.com/kunegis/pdfs/blob/master/kunegis:web-observatories.presentation.pdf}{\faSlideshare}
    \href{https://www.youtube.com/watch?v=L6AeRrunLmQ}{\faVideoCamera}}
    \href{http://wwsss16.webscience.org/speakers-and-tutors/dr-jerome-kunegis}{Web Science in Practice:  Web Observatories.}  
    Jérôme Kunegis, Steffen Staab, WSTNet Web Sci.\ Summer Sch., 2016.  
\end{itemize}

\section{Other Talks}
\begin{itemize}
  \item[{[\talknumber]}]
    Prediction of Network Semantics from Network Structure: Approaches
    Based on Large Network Collections.  Univ.\ of Mons, 2018. 
  \item[{[\talknumber]}]
    \marginnote{\href{https://www.slideshare.net/kunegis/title-what-is-the-difference-between-a-social-and-a-hyperlink-network-how-the-type-of-network-can-be-determined-from-the-network-structure-alone}{\faSlideshare}}
    \href{https://www.slideshare.net/kunegis/title-what-is-the-difference-between-a-social-and-a-hyperlink-network-how-the-type-of-network-can-be-determined-from-the-network-structure-alone}{What Is the Difference between a Social and a Hyperlink Network?
    -- How the Type of Network Can Be Determined from the Network Structure
    Alone}. Workshop on Network Comparison, 
    Univ.\ of Oxford, 2017.   
  \item[{[\talknumber]}]
    \marginnote{\href{https://www.slideshare.net/kunegis/measuring-the-conflict-in-a-social-network-with-friends-and-foes-a-recent-algebraic-approach}{\faSlideshare}}
    \href{https://www.slideshare.net/kunegis/measuring-the-conflict-in-a-social-network-with-friends-and-foes-a-recent-algebraic-approach}{Measuring the Conflict in a Social Network with Friends and Foes:  A
    Recent Algebraic Approach.}
    Tech.\ Univ.\ Berlin, 2017. 
  \item[{[\talknumber]}] 
    \marginnote{\href{https://github.com/kunegis/pdfs/blob/master/kunegis:konect-namur.presentation.pdf}{\faSlideshare}}
    \href{https://github.com/kunegis/pdfs/blob/master/kunegis:konect-namur.presentation.pdf}{The
      New KONECT Project at the University of Namur.}
    Univ.\ of Namur, 2017. 
  \item[{[\talknumber]}] 
    \marginnote{\href{https://github.com/kunegis/pdfs/blob/master/kunegis:graph-generator-namur.presentation.pdf}{\faSlideshare}}
    \href{https://github.com/kunegis/pdfs/blob/master/kunegis:graph-generator-namur.presentation.pdf}{Generating
      Networks with Realistic Properties Based on a 
    Given (Set of) Network(s), and a Short Overview of the KONECT
    Project.}  Univ.\ of Namur, 2016. 
  \item[{[\talknumber]}]
    \marginnote{
      \href{https://www.slideshare.net/kunegis/science-slam-by-jrme-kunegis-icwsm-2016}{\faSlideshare}
      \href{https://sites.google.com/site/icwsmscienceslam/}{\faStar}
      \href{https://www.youtube.com/watch?v=OxwsmNp3LFw}{\faVideoCamera}}
    \href{https://www.slideshare.net/kunegis/science-slam-by-jrme-kunegis-icwsm-2016}{It's Really about Hamsters.}
    Int.\ Conf.\ on Weblogs and Soc.\ Media Sci.\ Slam, 2016.  (Won First Prize)
  \item[{[\talknumber]}] 
    \marginnote{\href{https://github.com/kunegis/pdfs/blob/master/kunegis:konect-eth.presentation.pdf}{\faSlideshare}}
    \href{https://github.com/kunegis/pdfs/blob/master/kunegis:konect-eth.presentation.pdf}{KONECT:
      The Koblenz Network Collection -- Towards a Broad Analysis of Complex Systems.}  ETH Zürich, 2015.
  \item[{[\talknumber]}] 
    \marginnote{\href{https://github.com/kunegis/pdfs/blob/master/kunegis:shrinking-diversity.presentation.pdf}{\faSlideshare}}
    \href{https://github.com/kunegis/pdfs/blob/master/kunegis:shrinking-diversity.presentation.pdf}{Modeling
      the Evolution of Networks as Shrinking 
      Structural Diversity.}  Univ.\ of Koblenz--Landau, 2015. 
  \item[{[\talknumber]}]
    \marginnote{\href{https://github.com/kunegis/pdfs/blob/master/kunegis:konect-komepol.presentation.pdf}{\faSlideshare}}
    \href{https://github.com/kunegis/pdfs/blob/master/kunegis:konect-komepol.presentation.pdf}{Trust in Networks.}
    KoMePol Project, 2013. 
  \item[{[\talknumber]}] 
    \marginnote{\href{https://github.com/kunegis/pdfs/blob/master/kunegis:konect-bournemouth.presentation.pdf}{\faSlideshare}}
    \href{https://github.com/kunegis/pdfs/blob/master/kunegis:konect-bournemouth.presentation.pdf}{Observing
      the Web: The Koblenz Network Collection.}  Bournemouth Univ., 2013.
  \item[{[\talknumber]}] 
    \marginnote{\href{https://www.slideshare.net/kunegis/eight-ways}{\faSlideshare}}
    \href{https://www.slideshare.net/kunegis/eight-ways}{Eight
      Formalisms for Defining Graph Models.}  Univ.\ of Koblenz--Landau,
    2013.  
  \item[{[\talknumber]}]
    \marginnote{\href{https://github.com/kunegis/pdfs/blob/master/kunegis:diversity-uniformity.presentation.pdf}{\faSlideshare}}
    \href{https://github.com/kunegis/pdfs/blob/master/kunegis:diversity-uniformity.presentation.pdf}{Diversity
      vs Uniformity:  Understanding the Evolution of Large Networks.}
    ROBUST Project, 2012. 
  \item[{[\talknumber]}] 
    \marginnote{\href{https://github.com/kunegis/pdfs/blob/master/kunegis:modeling-linguistic-networks.presentation.pdf}{\faSlideshare}}
    \href{https://github.com/kunegis/pdfs/blob/master/kunegis:modeling-linguistic-networks.presentation.pdf}{Linguistic
      Network Analysis with the Koblenz Network Collection.}  
    Workshop on Modeling Linguist.\ Networks, 2012. 
  \item[{[\talknumber]}] Growth and Decay: Using Machine Learning to Predict the Web's
    Future, Workshop on Artif.\ Intell.\ on the Web, 2012. 
  \item[{[\talknumber]}] Models of Like, Dislike, Similarity and Dissimilarity using
    Split-complex Numbers, Univ.\ Coll.\ Dublin, 2012. 
  \item[{[\talknumber]}] 
    \marginnote{\href{https://www.slideshare.net/kunegis/why-beyonc-is-more-popular-than-me-fairness-diversity-and-other-measures}{\faSlideshare}}
    \href{https://www.slideshare.net/kunegis/why-beyonc-is-more-popular-than-me-fairness-diversity-and-other-measures}{Why Is Beyoncé More Popular Than Me:  Fairness, Diversity and
    Other Measures of Network Equality.} Univ.\ of Freiburg, 2012. 
  \item[{[\talknumber]}] Fairness on the Web: Alternatives to the Power
    Law. 
    Leibniz-Institut für Sozialwissenschaften, Cologne, 2012.  
  \item[{[\talknumber]}] 
    \marginnote{\href{https://github.com/kunegis/pdfs/blob/master/kunegis:network-survey.presentation.pdf}{\faSlideshare}}
    \href{https://github.com/kunegis/pdfs/blob/master/kunegis:network-survey.presentation.pdf}{KONECT
      -- The Koblenz Network Collection.}  WSTNet Meeting, 2012.  
  \item[{[\talknumber]}] On the Spectral Evolution of Large Networks.  Univ.\ Coll.\ Cork, 2011.   
  \item[{[\talknumber]}] 
    \marginnote{\href{https://github.com/kunegis/pdfs/blob/master/kunegis:explicit-diagonality.presentation.pdf}{\faSlideshare}}
    \href{https://github.com/kunegis/pdfs/blob/master/kunegis:explicit-diagonality.presentation.pdf}{Spectral
      Analysis of the Vector Space Model (and Explicit Semantic Analysis).}
    Jérôme Kunegis, Thomas Gottron, Univ.\ of Koblenz--Landau, 2011. 
  \item[{[\talknumber]}] On the Spectral Evolution of Large Networks.  Fraunhofer IAIS,
    St.\ Augustin, 2011.  
  \item[{[\talknumber]}] On the Spectral Evolution of Large Networks. Tech.\ Univ.\ Berlin, 2011.  
  \item[{[\talknumber]}] The Slashdot Zoo: Mining a Social Network with Negative
    Edges. Data and Knowl.\ Engineering Research Colloquium,
    Otto von Guericke Univ.\ Magdeburg, 2010.  
  \item[{[\talknumber]}] The Slashdot Zoo: Mining a Social Network with Negative
    Edges. Univ.\ of Koblenz--Landau, 2010. 
  \item[{[\talknumber]}] The Slashdot Zoo: Mining a Social Network with Negative
    Edges. Univ.\ of Hannover, 2010. 
  \item[{[\talknumber]}]
    \marginnote{\href{https://github.com/kunegis/pdfs/blob/master/kunegis:kernels-club.presentation.pdf}{\faSlideshare}}
    \href{https://github.com/kunegis/pdfs/blob/master/kunegis:kernels-club.presentation.pdf}{Kernel Methods.}
    Competence Center for Information Retrieval and Machine Learning (CC~IRML) Journal Club, DAI Laboratory, 2008. 
  \item[{[\talknumber]}] 
    \marginnote{\href{https://github.com/kunegis/pdfs/blob/master/hahne:gradualisiertes-communitymodell.presentation.pdf}{\faSlideshare}}
    \href{https://github.com/kunegis/pdfs/blob/master/hahne:gradualisiertes-communitymodell.presentation.pdf}{PIA+COMM~--
      an Intelligent Search Engine.} 
    Michael Hahne, Corinna Jung, Jérôme Kunegis, Andreas Lommatzsch and
    André Paus. 
    Workshop on the Formation of Soc.\ Netw.\ in Soc.\ Software
    Applications, INFORMATIK, 2006. 
  \item[{[\talknumber]}]
    \marginnote{\href{https://github.com/kunegis/pdfs/blob/master/kunegis:research-dai.presentation.pdf}{\faSlideshare}}
    \href{https://github.com/kunegis/pdfs/blob/master/kunegis:research-dai.presentation.pdf}{Research at the DAI Laboratory.}
    2006.
  \item[{[\talknumber]}]
    \marginnote{\href{https://www.slideshare.net/kunegis/schach-und-computer}{\faSlideshare}}
    \href{https://www.slideshare.net/kunegis/schach-und-computer}{Schach und Computer.}
    Seminar \emph{Geschichte der Entwicklung des Computers}, Tech.\ Univ.\ Berlin, 2005.
  \item[{[\talknumber]}]
    \marginnote{\href{https://github.com/kunegis/pdfs/blob/master/kunegis:kasparow-deep-blue.presentation.pdf}{\faSlideshare}}
    \href{https://github.com/kunegis/pdfs/blob/master/kunegis:kasparow-deep-blue.presentation.pdf}{Wettkampf Kasparow~--\ Deep Blue.}
    Ahmad Haj Fares and Jérôme Kunegis.
    Seminar \emph{Computerschach}, 2003. 
\end{itemize}

\section{Posters without Publication}
\begin{itemize}
  \item[{[P1]}] 
    \marginnote{\href{https://www.hashdoc.com/documents/174797/what-s-your-local-lingua-franca-quantifying-cultural-similarity-through-wikipedia-activity}{\faFilePictureO}
      \href{https://github.com/kunegis/pdfs/blob/master/samoilenko:netscix2016.jpg}{\faStar}
    }
    Linguistic Neighbourhoods:  Explaining Cultural Borders on
    Wikipedia through Multilingual Co-editing Activity.  Anna
    Samoilenko, Fariba Karimi, Daniel Edler, Jérôme Kunegis, Markus
    Strohmaier, Int.\ Sch.\ and Conf.\ on Netw.\ Sci.\ (NetSciX),
    2016.  (Won Best Poster Award)
  \item[{[P2]}]
    \marginnote{\href{https://github.com/kunegis/pdfs/blob/master/kunegis:stu-poster.pdf}{\faFilePictureO}}
    Do You Use Stu? Soon, You'll Do.
    Jérôme Kunegis, 
    WeST Institute Off-Campus Meeting (OCM), 2016. 
  \item[{[P3]}]
    \marginnote{\href{https://github.com/kunegis/pdfs/blob/master/kunegis:social-network-observatory.poster.pdf}{\faFilePictureO}}
    Social Network Observatory.  Jérôme Kunegis, Markus Strohmaier,
    Steffen Staab.  Computat.\ Soc.\ Sci.\ Winter Symp.\ (CSSWS), 2015.
  \item[{[P4]}] 
    Quantifying Cultural Similarity through Language Co-occurrences
    in Wikipedia Editing Activity. Anna Samoilenko, Fariba Karimi,
    Daniel Edler, Jérôme Kunegis, Markus Strohmaier.  Comput.\ Soc.\ Sci.\ Winter Symp.\ (CSSWS), 2015.
  \item[{[P5]}] 
    Polarisation in Voting Platforms: A Case Study of LiquidFeedback
    in the German Pirate Party. Manuel Mittler, Christoph Carl Kling,
    Jérôme Kunegis, Markus Strohmaier, Comput.\ Soc.\ Sci.\ Winter Symp.\ (CSSWS), 2015.
  \item[{[P6]}]
    Twitter as a Political Network -- Predicting the Following and
    Unfollowing Behavior of {German} Politicians. Julia Perl, Claudia
    Wagner, Jérôme Kunegis, Steffen Staab, Web Sci.\ Conf., 2015. 
  \item[{[P7]}] 
    Online Delegative Democracy: A Case Study of the German Pirate
    Party's Voting Platform.  Christoph Carl Kling, Jérôme Kunegis,
    Heinrich Hartmann.  Comput.\ Soc.\ Sci.\ Winter
    Symp.\ (CSSWS), 2014.
  \item[{[P8]}]
    Social Networking By Proxy: A Case Study of Catster, Dogster and
    Hamsterster. Comput.\ Soc.\ Sci.\ Winter
    Symp.\ (CSSWS), 2014.
  \item[{[P9]}] 
    A Theory-Driven Approach for Link and Unlink Predictions in
    Directed Social Networks. Julia Perl, Claudia Wagner, Jérôme
    Kunegis, Steffen Staab.  Comput.\ Soc.\ Sci.\ Winter
    Symp.\ (CSSWS), 2014.
  \item[{[P10]}]
    \marginnote{\href{https://github.com/kunegis/pdfs/blob/master/kunegis:konect-poster-2013.pdf}{\faFilePictureO}}
    Large Scale Network Analysis [KONECT -- The Koblenz Network Collection].
    Jérôme Kunegis,
    Tag der Computervisualistik (CV-Tag), University of Koblenz--Landau, 2013. 
  \item[{[P11]}]
    Need Networks? KONECT -- The Koblenz Network Collection.
    Eur.\ Summer Sch.\ on Inf.\ Retrieval (ESSIR), 2011. 
  \item[{[P12]}]
    KONECT -- The Koblenz Network Collection. Web Sci.\ Conf., 2011. 
  \item[{[P13]}]
    Uncovering Multi-modal Spread Modes using Joint 
    Diagonalization in Dynamic Human Contact Networks. 
    Damien Fay, Jérôme Kunegis, Eiko Yoneki.  Interdiscip.\ Workshop on Inf.\ and Decision in Soc.\ Netw., 2011. 
  \item[{[P14]}]
    \marginnote{\href{https://github.com/kunegis/pdfs/blob/master/kunegis:phd.poster.pdf}{\faFilePictureO}}
    On the Spectral Evolution of Large Networks. Postersession 
    für Nachwuchswissenschaftler/innen, Univ.\ of Koblenz--Landau,
    2011. 
  \item[{[P15]}]
    \marginnote{\href{https://github.com/kunegis/pdfs/blob/master/kunegis:sdm2010.poster.pdf}{\faFilePictureO}}
    On the Spectral Evolution of Large Networks. SIAM 
    Conf.\ on Data Min.\ Doctoral Forum (SIAM SDM), 2010. 
\end{itemize}

\section{Demonstrations}
\begin{itemize}
 \item[{[D1]}] 
   \marginnote{\href{https://github.com/kunegis/pdfs/blob/master/kunegis:sit.presentation.pdf}{\faSlideshare}}
   Schülerinformationstage, Institute for Web Science and
   Technologies, Univ.\ of Koblenz--Landau, 2011, 2013, 2015, 2016.  
 \item[{[D2]}] Institute for Web Science and Technologies, conference booth.
   \href{http://informatik2013.de}{INFORMATIK 2013}.  
 \item[{[D3]}] 
   \marginnote{\href{https://livetweet.west.uni-koblenz.de/}{\faChain}
     \href{https://link.springer.com/chapter/10.1007/978-3-642-28997-2_66}{\faFileTextO}}
   LiveTweet: Monitoring and Predicting Interesting Microblog
       Posts, \href{http://ecir2012.upf.edu/}{Eur.\ Conf.\ on Inf.\ Retrieval} (ECIR),
       2012. 
 \item[{[D4]}] PIA, Spree, CeBIT, 2010.
 \item[{[D5]}] 
   \marginnote{\href{http://www.aot.tu-berlin.de/index.php?id=1540&tx_ttnews[tt_news]=725}{\faChain}}
   PIA+COMM, CeBIT, 2007.
 \item[{[D6]}] PIA, Lange Nacht der Wissenschaften Berlin/Potsdam, 2006. 
\end{itemize}

\section{Program Committees}
\begin{itemize}
\item \href{http://www.complexnetworks.org/}{Int.\ Conf.\ on Complex
  Netw.\ and Their Applications} [\emph{Workshop} in 2016], 2016, 2017, 2018.   
\item \href{http://www.icwsm.org/2016/index.php}{Int.\ AAAI Conf.\ on
  Web and Soc.\ Media} (ICWSM), 2016.
\item \href{http://www.snow-workshop.org/}{Workshop on Soc.\ News on
  the Web} (SNOW) at the World Wide Web Conf.\ (WWW), 2014, 2016. 
\item \href{http://events.kmi.open.ac.uk/salsa2016/}{Workshop on Soc.\ Semant.\ Anal.}\ (SALSA), 2016.
\item \href{https://failworkshops.wordpress.com/fail-at-ir16/}{\#FAIL!
  -- The Workshop Series} at the Internet Research Conf.\ (IR), 2015.
\item \href{http://www.www2015.it/call-for-web-science-track/}{Web
  Sci.\ Track}, Int.\ World Wide Web Conf.\ (WWW), 2014, 2015.
\item \href{https://failworkshops.wordpress.com/fail-workshop-at-websci15/}{\#FAIL! -- The Workshop Series} at the Web Sci.\ Conf.\ (WebSci), 2015.
\item \href{http://www.websci14.org/}{Web Sci.\ Conf.},
  2011, 2012, 2013, 2014. 
\item \href{http://www.cool2014.com/}{Workshop on Connecting Online \&
  Offline Life} (COOL) at the World Wide Web Conf.\ (WWW), 2014. 
\item \href{http://ijcai13.org/}{Int.\ Joint Conf.\ on Artif.\ Intell.} (IJCAI), 2013.  
\item \href{http://webscience-education-workshop.blogs.usj.edu.lb/}{Web
  Sci.\ Education Workshop} at the Web Sci.\ Conf., 2013. 
\item \href{http://www.ftsm.ukm.my/stair13/}{Conf.\ on Semant.\ Tech.\ and Inf.\ Retrieval} (STAIR), 2013. 
\item \href{http://spim-workshop.org/}{Int.\ Workshop on Semant.\ Personalized Inf.\ Manag.}\ (SPIM) at the Int.\ Conf.\ on Web
  Search and Data Min.\ (WSDM), 2013.  
\item \href{http://ecir2013.org/}{Eur.\ Conf.\ on Inf.\ Retrieval} (ECIR), 2013.  
\item \href{http://mama.west.uni-koblenz.de/}{Workshop on Metrics,
  Anal.\ and Tools for Online Community Manag.}\ (MAMA), at
  INFORMATIK, 2013
\item \href{http://ecir2012.upf.edu/}{Eur.\ Conf.\ on Inf.\ Retrieval} (ECIR), poster track, 2012.  
\item \href{http://www.oegai.at/konvens2012/}{Conf.\ on Natural Language
  Processing} (KONVENS), 2012. 
\item \href{http://spim-workshop.org/}{Workshop on Personalized
  Inf.\ Manag.: Linking Soc.\ and Semant.\ Web} (SPIM) at the
  Int.\ Conf.\ on Web Engineering (ICWE), 2012. 
\item \href{http://www.cikm2011.org/}{Int.\ Conf.\ on Inf.\ and Knowl.\ Manag.}\ (CIKM), 2011.
\item \href{http://www.sigkdd.org/kdd/2011/}{Int.\ Conf.\ on 
  Knowl.\ Discovery and Data Min.} (KDD), research track, 2011.
\item \href{http://www.dai-labor.de/camra2010/}{Challenge on
  Context-aware Movie Recomm.} (CAMRa) at the Conf.\ on  
  Recomm. Syst.\ (RecSys), 2010.   
\item \href{http://www.dai-labor.de/unm2010/}{Special Session on Uncertainty in Netw.\ Min.} (UNM) at the
  Int.\ Conf.\ on Inf.\ Processing and Manag.\ of
  Uncertainty in Knowl.-based Syst.\ (IPMU), 2010. 
\end{itemize}

\section{Other Reviewing}
\begin{itemize}
\item Royal Military Academy, Belgium, 2018. 
\item \href{http://journals.cambridge.org/action/displayJournal?jid=NWS}{Netw.\ Sci.}, 2015, 2016.
\item \href{http://surveys.acm.org/}{ACM Comput.\ Surveys}, 2014, 2015. 
\item \href{http://www.webscience-journal.net}{The J.\ of Web Sci.}, 2014, 2018. 
\item \href{http://toit.acm.org/}{ACM Trans.\ on Internet
  Tech.}\ (TOIT), 2014.  
\item \href{http://www.ii.pwr.wroc.pl/~krol/eng_EPP.htm}{Special
  Issue on ``Propagation Phenomenon in Complex Networks:  Theory and
  Practice''}, New Generation Comput., 2014. 
\item \href{http://iswc2013.semanticweb.org/}{Int.\ Semant.\ Web Conf.}
  (ISWC), 2013.  
\item \href{http://www.compstat2012.org/}{Int.\ Conf.\ on Comput.\ Stat.} (COMPSTAT), 2012.  
\item
  \href{http://www.journals.elsevier.com/computer-networks/}{Int.\ J.\ of
    Comput.\ and Telecommunications Netw.} (COMNET), 2012. 
\item \href{http://ieee-cis.org/pubs/tnn/}{IEEE Trans.\ on Neural
  Netw.} (TNN), 2011.   
\item
  \href{http://www.springer.com/computer/ai/journal/10458}{J.\ on Auton.\ Agents and Multi-agent Syst.}\ (AAMAS), Special Issue on Agent Mining,
  2011. 
\item \href{http://tkdd.cs.uiuc.edu/}{ACM Trans.\ on Knowl.\ Discovery from Data} (TKDD), 2011.  
\item \href{http://mklab.iti.gr/tvca/}{TV Content Anal.}\ (TVCA), 2011.
\item \href{http://www.loa-cnr.it/ontolex2010}{Workshop on Ontologies
  and Lexical Resources} (OntoLex) at the Int.\ Conf.\ on Comput.\ Linguist.\ (COLING), 2010. 
\end{itemize}

\section{Volunteering}
\begin{itemize}
  \item \href{http://wwsss16.webscience.org/}{WSTNet Web Science Summer School} (WWSSS), Koblenz, 2016. 
  \item \href{http://www.festivalofideas.cam.ac.uk/}{Festival of Ideas}, Cambridge, 2013. 
  \item \href{http://iswc2011.semanticweb.org/}{International Semantic Web Conference} (ISWC), Bonn, 2011
  \item \href{https://essir.uni-koblenz.de/}{European Summer School on Information Retrieval} (ESSIR), Koblenz, 2011. 
\end{itemize}

\section{Software Authorship}
\begin{itemize}
  \item
    \marginnote{\href{https://github.com/kunegis/stu}{\faCog} 
      \href{https://networkscience.wordpress.com/2018/01/15/the-build-system-stu-in-practice/}{\faComment}
      \href{https://github.com/kunegis/pdfs/blob/master/kunegis:stu-poster.pdf}{\faFilePictureO}}
    \href{https://github.com/kunegis/stu}{Stu}, build automation (C++14)
  \item 
    \marginnote{\href{https://github.com/kunegis/konect-toolbox}{\faCog}}
    \href{https://github.com/kunegis/konect-toolbox}{KONECT Toolbox}, network analysis toolbox (Matlab, C99)
  \item 
    \marginnote{\href{https://github.com/kunegis/polcovar}{\faCog} \href{http://arxiv.org/abs/1402.5835}{\faFileExcelO}}
    \href{https://github.com/kunegis/polcovar}{Polcovar}, counting subgraph patterns in random graphs (Matlab)
  \item
    \marginnote{\href{https://github.com/kunegis/universal-recommender}{\faCog}}
    \href{https://github.com/kunegis/universal-recommender}{Universal Recommender}, recommendation library (Java)
  \item
    \marginnote{\href{https://github.com/kunegis/babychess}{\faCog}}
    \href{https://github.com/kunegis/babychess}{BabyChess}, chess engine and GUI (C++98)
  \item \emph{More:  see {\tt \href{https://github.com/kunegis}{github.com/kunegis}}}
\end{itemize}

\section{Other}
\begin{itemize}
\item
  \marginnote{
    \href{https://apps.rhs.org.uk/horticulturaldatabase/orchidregister/orchiddetails.asp?ID=1067542}{\faLeaf}
    \href{https://orchidroots.com/display/summary/orchidaceae/101067542/}{\faChain}
    \href{https://orchidex.org/paphiopedilum/belgian-sunshine/1067542}{\faChain}
  }
  \textit{Paphiopedilum} Belgian Sunshine, first cultivation of hybrid orchid.
\end{itemize}

\section{Mentions in Media}
\begin{itemize}
  \item
    \marginnote{\href{https://www.piratenpartei.de/2015/05/31/die-abstimmungssoftware-liquidfeedback-der-piratenpartei-wegweisend-fuer-demokratie-2-0/}{\faNewspaperO}}
    \href{https://www.piratenpartei.de/2015/05/31/die-abstimmungssoftware-liquidfeedback-der-piratenpartei-wegweisend-fuer-demokratie-2-0/}{Die
      Abstimmungssoftware LiquidFeedback in der Piratenpartei:
      wegweisend für Demokratie 2.0?} (in German), www.piratenpartei.de, May 2015. 
\item
  \marginnote{\href{http://www.wired.it/attualita/2015/04/01/come-funzionato-davvero-democrazia-liquida-dei-pirati-tedeschi/}{\faNewspaperO}}
  \href{http://www.wired.it/attualita/2015/04/01/come-funzionato-davvero-democrazia-liquida-dei-pirati-tedeschi/}{Come
    ha funzionato davvero la democrazia liquida dei Pirati tedeschi} (in Italian),
  Wired Italia, April 2015. 
\end{itemize}

\section{Awards and Nominations}
\begin{itemize}
  \item
    \marginnote{
      \href{https://sites.google.com/site/icwsmscienceslam/}{\faStar}
      \href{https://www.slideshare.net/kunegis/science-slam-by-jrme-kunegis-icwsm-2016}{\faSlideshare}
      \href{https://www.youtube.com/watch?v=OxwsmNp3LFw}{\faVideoCamera}}
    First Prize, Int.\ Conf.\ on Web and Soc.\ Media (ICWSM) Science Slam, 2016. 
  \item 
    \marginnote{\href{https://twitter.com/mstrohm/status/688058691799625728}{\faStar}
      \href{https://www.hashdoc.com/documents/174797/what-s-your-local-lingua-franca-quantifying-cultural-similarity-through-wikipedia-activity}{\faFilePictureO}
    }
    Best Poster Award, Int.\ Sch.\ and Conf.\ on Netw.\ Sci.
    (NetSciX), 2016. 
  \item 
    \marginnote{\href{http://www.icwsm.org/2015/home/news/}{\faStarO}
      \href{http://www.aaai.org/ocs/index.php/ICWSM/ICWSM15/paper/view/10566}{\faFileTextO}
      \href{https://arxiv.org/pdf/1503.07723.pdf}{\faFileExcelO}
      \href{https://www.piratenpartei.de/2015/05/31/die-abstimmungssoftware-liquidfeedback-der-piratenpartei-wegweisend-fuer-demokratie-2-0/}{\faNewspaperO}
      \href{http://www.wired.it/attualita/2015/04/01/come-funzionato-davvero-democrazia-liquida-dei-pirati-tedeschi/}{\faNewspaperO}
    }
    Best Paper Honorable Mention, Int.\ Conf.\ on Web and Soc.\ Media (ICWSM), 2015.  
  \item 
    \marginnote{\href{http://www.sigweb.org/resources/sigweb-awards/43-ted-nelson}{\faStarO}
      \href{http://dl.acm.org/citation.cfm?id=2310039}{\faFileTextO}
      \href{https://pt.slideshare.net/kunegis/presentation-2012ht}{\faSlideshare}
    }
    Ted Nelson Newcomer Award Nominee, Conf.\ on Hypertext and
    Soc.\ Media (HT), 2012.   
  \item 
    \marginnote{\href{http://www.yorku.ca/cikm10/awards.php}{\faStar}
      \href{http://dl.acm.org/citation.cfm?id=1871533}{\faFileTextO}
      \href{https://www.slideshare.net/kunegis/presentation-5758116}{\faSlideshare}
    }
    Student Travel Award, Int.\ Conf.\ on Inf.\ and Knowl.\ Manag.\ (CIKM), 2010. 
  \item 
    \marginnote{\href{http://www.siam.org/prizes/sponsored/travel.php}{\faStar}}
    Student Travel Fellowship Award, SIAM Conf.\ on Data Min.\ (SDM),
    2010.  
  \item 
    \marginnote{\href{http://www.data-mining-forum.de/paper_award_2007.php}{\faStarO}
      \href{https://www.semanticscholar.org/paper/Collaborative-Filtering-using-Electrical-Resistanc-Kunegis-Schmidt/8266ddfc2cbfb6d69017ba7696bfe5335ea11b21}{\faFileTextO}
    }
    Best Paper Nominee, Industrial Conf.\ on Data Min.\ (IndCDM), 2007.    
  \item 
    \marginnote{\href{https://mathinfo.unistra.fr/irem/rallye-mathematique-dalsace/sujets-corriges-et-palmares/}{\faStar}}
    First Prize, Rallye mathématique d'Alsace, Terminale (Alsace
    Mathematical Rally, 12\textsuperscript{th} grade), 1999. 
\end{itemize}

\section{Popularization}     \input{list-popularization}

\section{Skills}
\begin{itemize}
  \item \textbf{Languages:}
    French (native), 
    German (native),
    English (advanced, C2), 
    Dutch (beginner, A2), 
    Spanish (intermediate), 
    Portuguese (beginner).  
  \item \textbf{Programming languages:}
    Major work performed in:  C (C90, C99), C++ (experience with C++98, C++14, C++17,
    C++20), Java (SE~6, SE~8), Matlab (2009), Perl~(5), Python~(3), shell
    scripting (POSIX, Bash, KSH); work performed in:
    Ada, AWK, BASIC, JavaScript, Julia, Logo, Object Pascal, Octave, PHP, Turbo Pascal, Tcl.
  \item \textbf{Operating systems:}
    Major work performed on: GNU/Linux (Ubuntu, Red Hat), Windows, AIX; work
    performed on: FreeBSD, macOS.  
  \item \textbf{Source code management and bug tracking:}
    Bugzilla, CVS, Git, Gitlab/Github/Bitbucket, Jira, Kanban, Subversion/SVN. 
  \item \textbf{Build systems and continuous integration/delivery (CI/CD):}
    Ant, Autotools/Autoconf/Automake, CMake, Conan, Cook, Jenkins, Make, Maven, Stu (I am the
    author).
  \item \textbf{Programming environments and editors:}
    Eclipse, Emacs, IntelliJ, vi.
  \item \textbf{Query, data processing and representation technologies:}
    CSV/TSV, RDF, relational databases, sed, SQL, XSLT. 
  \item \textbf{Typesetting and layouting:}
    CSS, HTML, JSP, Tex/Latex. 
  \item \textbf{Programming methodologies and topics:}
    Agent-oriented programming,
    agile software development,
    character encodings,
    chess programming,
    data manipulation~/ data engineering,
    internationalization,
    linear programming~/ optimization,
    machine learning,
    message passing,
    numerical computing,
    object oriented programming,
    parsing, 
    regular expressions,
    scrum,
    systems programming, visualization~/ plotting. 
  \item \textbf{Networking and organizatorial skills:}
    Organisation of conferences, workshops, off-campus meetings,
    use of Twitter/Facebook for outreach, popularization of scientific topics. 
  \item \textbf{Teaching and didactical skills:}
    Lectures, seminars, projects, supervision of bachelor/master/PhD theses,
    drafting and correcting exams.
    I~have given both lectures and academic talks in English, French and
    German.
  \item \textbf{Scientific topics:}
    Algebraic graph theory,
    complex networks,
    data mining,
    databases~/ relational algebra,
    game theory,
    graph theory,
    linear algebra,
    machine learning,
    matrix decompositions,
    network science, network theory,
    random graphs,
    recommender systems,
    search engines~/ information retrieval,
    semantic web,
    web science.
  \item \textbf{Research highlights:}
    First study and large published dataset about social networks with negative `enemy'
    edges~\citek{kunegis:slashdot-zoo}, 
    large-scale study of web tracking~\citek{schelter:trackers},
    algebraic analysis of bipartite networks~\citek{kunegis:hyperbolic-sine} and
    signed graphs~\citek{kunegis:signed-kernels},
    invented a machine learning technique for link
    prediction~\citek{kunegis:spectral-transformation},
    introduced the spectral evolution model for networks~\citek{kunegis:spectral-network-evolution},
    introduced measures for the skewness of degree distributions~\citek{kunegis:power-law}.
\end{itemize}

\iffalse
\section{Soft Skills}
\begin{itemize}
\item \textbf{Adaptability:}
  I have worked both in the industry and in academia.  This gives me a unique
  perspective on the value of programming and management techniques, as well as
  the experience when to use them. 
\item \textbf{Communication:}
  I love to convey ideas to other people.  As a scientist, talking about new
  approaches to problems is second nature to me.  As is disseminating results to
  larger audiences. 
\item \textbf{Teamwork:}
  I am able to work in teams small and large, with collaborators close and
  distant, from my own teams as well as from other companies.  I have both
  managed a team of developers, as well as been an individual developer in a team.
\item \textbf{Creativity:}
  My usual approach is to go beyond the standard ``do what everyone does'' and
  to find novel solutions to problems.  This is a core skill for a scientist,
  which I also strive to apply to software development. 
\item \textbf{Analytic thinking:}
  Being precise is important to me.  ``It kind of works'' is not enough~--
  things must be shown to work via extensive testing, as well as via theoretical
  consideration. 
\item \textbf{Eclecticism:}
  There's more than one way to do it.  I don't believe that there is one ideal
  way to solve a given problem.  In my career, I have learned and used a large
  diversity of programming languages in order to give me a broad range of tools
  to solve real-world computer science problems.
\end{itemize}
\fi

\section{Software Development}
\begin{itemize}
\item \textbf{Swift's Solution for CREST (CREST over SwiftNet)} \hfill 2023--present \\
  Message passing platform (C, C++)
\item \textbf{Tracking Analysis} \hfill 2018 \\
  Analysis of large-scale web tracking data (shell script, Matlab)
\item \textbf{Santé} \hfill 2018--present \\
  Health data mining and analysis (Python)
\item \textbf{Social Information Processing} \hfill 2014--2017 \\
  Cloud infrastructure for intra-department collaboration on computing projects
  (acquisition and management of 18-machine cloud)
\item \textbf{REVEAL} \hfill 2014--2016 \\
  Latent data discovery (Octave)
\item \textbf{LiquidFeedback} \hfill 2014--2015 \\
  Extraction and analysis of data from a liquid democracy platform (Python,
  shell scripting)
\item \textbf{Polcovar} \hfill 2013 \\
  Symbolic calculation of the number of arbitrary subgraph counts in Erdős--Rényi
  graphs (Matlab, Octave)
\item \textbf{SynGraphy} \hfill 2013--2015 \\
  A library for graph generation based on arbitrary graph properties (Matlab) 
\item \textbf{Petster} \hfill 2013--2017 \\
  Crawling and analysis of social media data from pet social networks (shell
  scripting, Matlab, Python)
\item \textbf{SocialSensor} \hfill 2013--2014 \\
  Social media content analysis platform (Java)
\item \textbf{KONECT-Toolbox} \hfill 2012--2022 \\
  Toolbox for calculation of network statistics, matrix decompositions and other
  characteristics (Matlab, Octave)
\item \textbf{Stu} \hfill 2012--present \\
  Build system for large data analysis projects (C++14)
\item \textbf{KONECT-Web} \hfill 2011--present \\
  Website of the KONECT project (HTML, shell scripting, Perl, Octave, Stu)
\item \textbf{KONECT-Analysis} \hfill 2011--2022 \\
  Analysis code of KONECT, creation of plots and visualizations (C++11, shell
  scripting, Matlab, Octave, Julia, Perl)
\item \textbf{KONECT-Extr} \hfill 2010--2019 \\
  Extraction library for the 1000+ KONECT network datasets (shell scripting, C99,
  Perl)
\item \textbf{ROBUST} \hfill 2010--2013 \\
  Analysis of business communities (Java)
\item \textbf{LiveTweet} \hfill 2010--2011 \\
  Extraction and analysis of Twitter trends (Java, Matlab)
\item \textbf{Learning Semantic Recommenders} \hfill 2010 \\
  Link prediction library on network data with semantic relationships (Java)
\item \textbf{SmartSenior} \hfill 2009--2010 \\
  Intelligent services for elderly people (Java)
\item \textbf{WebTV} \hfill 2008--2010 \\
  Recommender system for smart TVs (Java)
\item \textbf{Slashdot Zoo} \hfill 2008--2009 \\
  Crawling Slashdot.org and analysis of friend/foe relationshis (shell
  scripting, Matlab)
\item \textbf{Universal Recommender} \hfill 2007--2010 \\
  Library for building recommender systems (Java)
\item \textbf{Search Results Optimization via Linear Integer Programming} \hfill 2006 \\
  Search engine using linear integer programming to support hard and soft
  constraints (Java)
\item \textbf{DDDC} \hfill 2005--2010 \\
  Monitoring tool for continuous deployment (shell scripting)
\item \textbf{PIA -- Personal Information Agent} \hfill 2002--2010 \\
  Search engine and social exchange site for scientific documents (Java, HTML,
  CSS)
\item \textbf{BabyChess} \hfill 1998--2006 \\
  Chess engine and desktop application (C++98)
\end{itemize}

\section{Miscellaneous}
\begin{itemize}
  \item Birthday:  March 27
  \item Full titles:  Dr.\ rer.\ nat.\ Dipl.-Inform.\ Jérôme KUNEGIS
\end{itemize}

\section{Legend}
\begin{itemize}
  \item[]
    \faFileTextO~Publication is available ~~
    \faFileExcelO~Publication is available on arXiv ~~
    \faFilePictureO~Poster is available ~~
    \faSlideshare~Slides are available ~~
    \faQuoteRight~Publication has 100+ citations ~~
    \faChain~Publication/event has an accompanying website ~~
    \faDatabase~Dataset is available ~~
    \faCog~Source code is available ~~
    \faVideoCamera~Video of talk is available ~~
    \faComment~A blog article was written about the topic ~~
    \faLeaf~Official registration of hybrid orchid ~~
    \faNewspaperO~Publication has media coverage ~~
    \faStar~I won a prize for this work ~~
    \faStarO~Nomination for a prize 
\end{itemize}

\end{resume}
\end{document}
